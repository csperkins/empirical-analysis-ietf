\documentclass[twocolumn,10pt]{article}
\usepackage[l2tabu,orthodox]{nag}
\usepackage[utf8x]{inputenc}
\usepackage[british]{babel}
\usepackage{microtype}
\usepackage{amsmath}
\usepackage[all]{onlyamsmath}
\usepackage{newtxtext}
\usepackage{newtxmath}
\usepackage[caption=false]{subfig}
\usepackage{booktabs}
\usepackage{upquote}
\usepackage{graphicx}
\usepackage{url}
\usepackage{algorithm}
\usepackage{algpseudocode}
\usepackage{color}
\usepackage[cm]{fullpage}

\frenchspacing
\uchyph=0

\newcommand{\todo}[1]{\textbf{\textcolor{red}{To do: #1}}}
\newcommand{\pb}[1]{\vspace{0.75ex}\noindent{\textbf{#1}}}

%==================================================================================================
\begin{document}

\title{An Empirical Analysis of the Internet Engineering Task Force 
       with Computational Methods}
\author{Colin Perkins\\University of Glasgow}
\maketitle
%==================================================================================================
\begin{abstract}
  % Four sentences:
  %  - State the problem
  %  - Say why it's an interesting problem
  %  - Say what your solution achieves
  %  - Say what follows from your solution

  The Internet requires interoperability between networks, systems, and
  applications, as well as cooperation among a growing number of
  stakeholders. The Internet Engineering Task Force (IETF) is critical in
  supporting this cooperation and interoperability by bringing together
  interested parties and standardising the protocols that power the
  Internet, such as IPv4/6 or HTTP/s. Our work covers more 20 years of IETF
  data, and analyses the participants of the IETF, the emails they
  exchange, and the documents they produce. We show how these data can be
  used to understand who are these Internet actors, and their interests as
  well as how Internet's growth and maturity has given rise to a longer and
  complex standardisation process where participants gain and exercise
  influence and how this is reflected in the language they use and the
  interactions they have.
  

\end{abstract}
%==================================================================================================
\section{Introduction}

% A good paper introduction is fairly formulaic. If you follow a simple set
% of rules, you can write a very good introduction. The following outline can
% be varied. For example, you can use two paragraphs instead of one, or you
% can place more emphasis on one aspect of the intro than another. But in all
% cases, all of the points below need to be covered in an introduction, and
% in most papers, you don't need to cover anything more in an introduction.
%
% Paragraph 1: Motivation. At a high level, what is the problem area you
% are working in and why is it important? It is important to set the larger
% context here. Why is the problem of interest and importance to the larger
% community?



% Paragraph 2: What is the specific problem considered in this paper? This
% paragraph narrows down the topic area of the paper. In the first
% paragraph you have established general context and importance. Here you
% establish specific context and background.



% Paragraph 3: "In this paper, we show that...". This is the key paragraph
% in the introduction - you summarize, in one paragraph, what are the main
% contributions of your paper, given the context established in paragraphs
% 1 and 2. What's the general approach taken? Why are the specific results
% significant? The story is not what you did, but rather:
%  - what you show, new ideas, new insights
%  - why interesting, important?
% State your contributions: these drive the entire paper.  Contributions
% should be refutable claims, not vague generic statements.

In this paper, we ...

% Paragraph 4: What are the differences between your work, and what others
% have done? Keep this at a high level, as you can refer to future sections
% where specific details and differences will be given, but it is important
% for the reader to know what is new about this work compared to other work
% in the area.



% Paragraph 5: "We structure the remainder of this paper as follows." Give
% the reader a road-map for the rest of the paper. Try to avoid redundant
% phrasing, "In Section 2, In section 3, ..., In Section 4, ... ", etc.

We structure the remainder of this paper as follows.

%==================================================================================================
\section{Background and Datasets}

% The following is from our IMC 2021 paper:

We start by presenting an overview of the IETF, and the publication process
for RFCs, before outlining the data sources we use within this paper. We
also highlight the ethical considerations of accessing and processing this
data.


%--------------------------------------------------------------------------------------------------
\subsection{An IETF Primer}

% The following is from our IMC 2021 paper:

%..................................................................................................
\pb{The IETF} is an open standards organisation, which develops
Internet standards via contributions and collaborations across a number of
voluntary stakeholders, including academics, consultants, industry
representatives, governments, and civil society organisations.  Through
extensive collaboration across contributors, the IETF, and associated
organisations, develops Internet standards and other documents. These are
published by the RFC Editor (\url{https://www.rfc-editor.org}) in four
\emph{publication streams}: the IETF stream, the Internet Research Task
Force (IRTF) stream, the Internet Architecture Board (IAB) stream, and the
Independent Submission stream.

There is also a fifth, legacy stream, comprising RFCs published prior to
the adoption of separate publication streams in July 2007 \cite{rfc4844}.
While the IETF is an open standards forum that develops technical standards
and operational guidelines for the Internet, the IRTF is an associated
organisation that promotes longer-term research, and the IAB provides
long-range technical direction for Internet development. The Independent
Submission stream ``allows RFC publication for some documents that are
outside the official IETF/IAB/IRTF process but are relevant to the Internet
community'' (\url{https://www.rfc-editor.org/about/independent}).

%..................................................................................................
\pb{The standards development process} is an inherently collaborative
activity.  Most day-to-day work is conducted on public mailing lists, in
conjunction with three plenary meetings and numerous interim working group
meetings per year. 
The mailing lists are broadly split into three categories: announcement
lists, where replies are not allowed; non-working group lists, for
discussing topics that do not relate to the work within an IETF working
group or IRTF research group; and working group and area lists, where
technical discussions take place.

The process of RFC publication begins with the submission of an
\emph{Internet-Draft}. Whereas anyone can post a draft, not all drafts
become RFCs. After a draft is first posted, multiple revisions might take
place resulting in multiple versions of the draft. Each new draft is
announced on one or more mailing lists related to the topic of the draft,
soliciting feedback and encouraging discussion. Drafts are initially posted
by individuals. For publication under the IETF stream, drafts must then be
adopted by a working group, where, via further revision, the draft may
ultimately be published as an RFC. The process of managing drafts, and the
degree and type of peer review conducted prior to their submission to the
RFC Editor for publication, differs between streams.

Once the technical development of the draft is complete its publication is
managed by the RFC Editor, who maintains the master archive of the RFC
documents, along with an index of metadata pertaining to the RFCs and their
authors. Finally, once an RFC has been published, deployment is voluntary,
and therefore not all RFCs are widely implemented.

%--------------------------------------------------------------------------------------------------
\subsection{Data Sources}


%--------------------------------------------------------------------------------------------------
\subsection{Ethical Considerations}
\label{sec:ethics}

% The following is from our IMC 2021 paper:

The data we analyse is extracted from public IETF archives and APIs.  We
have taken steps to ensure ethical compliance.  To ensure that our access
to these services does not cause operational problems for the IETF, we are
in regular contact with the IETF Tools Team and Secretariat, as well as the
operators of the Datatracker and mailing list archive.  We have extensively
discussed our work with IETF leadership (IETF and IAB Chairs, the IETF
Executive Director, and the IRTF Chair, who is a co-author on this paper)
to ensure that our access falls within their acceptable use policies.

Participation in the IETF is dependent on agreement to abide by the
policies and procedures described at \url{https://www.ietf.org/about/note-well},
including the privacy policy at \url{https://www.ietf.org/privacy-statement}.
These make explicit provision that mailing list archives and the metadata
contained in the Datatracker system will be made public, and it is this
public data that we process to extract the aggregate statistics presented
in Sections \ref{sec:documents}, \ref{sec:authors}, and \ref{sec:mail}, as
well as the features analysed in Section \ref{sec:model}. We transfer and
store data securely, and retain it only for the time needed to perform the
analysis.  Since we operate entirely using the public APIs provided by
IETF, we have no access to private data about individuals.

In balancing these ethical considerations with the reproducibility of our
work, we provide the tools needed to access the datasets from the relevant
IETF sources, rather than the data itself.


%==================================================================================================
\section{Trends in RFC Publication}

% \begin{figure}
%   \centering
%   \includegraphics{figures/rfcs-by-year-stream.pdf}
%   \caption{Number of RFCs published per year}
%   \label{fig:rfcs-by-year}
% \end{figure}

%--------------------------------------------------------------------------------------------------
\subsection{RFC Production and Complexity}

RFC production trends show clear evidence that the IETF is a mature
standards development organisation. The peak of activity was in the 1990s
and early 2000s, with the initial broad deployment of the Internet, and the
organisation has since entered more of a development and maintenance mode
of operation. There are clear signs that the complexity of the installed
base of protocols is slowing development of new technologies.

% The following is from our IMC 2021 paper, Section 3.1:

%..................................................................................................
\pb{RFC Publication Rate:}
In total, 8,711 RFCs have been published through to the end of 2020.
Figure~\ref{fig:rfcs_by_area} shows how publication trends, in terms of
IETF areas and non-IETF streams, have changed over time. We identify three
broad publication phases in the RFC series. First, in 1969 through 1974,
RFCs are published at a rapid rate during the initial development of the
ARPANET. Then, from 1975 through 1985, development slows. This reflects a
relatively small community that is gaining real-world experience with the
network and developing a small core of applications and protocols. Finally,
with the creation of the IETF and the introduction of the National Science
Foundation Network (NSFNET) in 1986, both the community, and the number of
RFCs published, starts to expand rapidly. This is further driven by the
opening of the network to commercial and public use in the mid-1990s, and
continues to this day.

As shown, the annual RFC publication rate was highest in 2005, at the peak
of the standardisation efforts for SIP and related standards for voice-over-IP
and Internet telephony. The rate of publication has slowed in recent years,
following the completion of large work programmes relating to HTTP/2 and
WebRTC.

\begin{figure}
\includegraphics[width=0.45\textwidth]{figures-prev/imc-2021/documents/rfcs_areas.pdf}
\caption{RFCs by area. \textit{Other} includes legacy RFCs, and RFCs from other non-IETF streams.}
\label{fig:rfcs_by_area}
\end{figure}


%..................................................................................................
\pb{Role of Working Groups:}
From the creation of the IETF in 1986, the growing community with its
interests in an increasing set of applications and protocols, has been
split into working groups. These working groups are chartered with a focus
on a well-defined programme of work, and exist within areas that have a
broader focus.  As shown in Figure~\ref{fig:rfcs_by_area}, the output of
different areas has remained relatively stable over time. The most notable
trends begin with the creation of the Real-time Applications and
Infrastructure (rai) area from within the Transport (tsv) area, and its
later merger with the Applications (app) area to become the Applications
and Real-Time (art) area around 2014. Additionally, we also observe the
significant growth in output of the Routing (rtg) area, owing to the
ongoing development of standards for MPLS, service function chaining, and
fat tree routing in data centres.

To give a sense for the broader productivity of the IETF,
Figure~\ref{fig:pub_wgs_yearly} shows the number of working groups that
publish RFCs each year. This highlights how the structure of the IETF has
grown to accommodate its larger community: in the early 1990s, fewer than
20 working groups were actively publishing RFCs, while in recent years
there has typically been at least 60 different publishing groups, with a
peak of 97 active working groups, and other activities, in 2011.

\begin{figure}
\includegraphics[width=0.45\textwidth]{figures-prev/imc-2021/documents/unique_wgs_per_year_areas.pdf}
\caption{Number of publishing working groups. \textit{Other} includes legacy RFCs, IRTF research groups, and non-IETF streams.}
\label{fig:pub_wgs_yearly}
\end{figure}

Figure~\ref{fig:rfcs_days_to_pub} plots the median number of days from the
submission of an RFC's first draft, through to its publication as an RFC.
This shows a clear trend: RFCs are taking longer to make their way through
the standardisation and publication process. The median number of days to
publication was 469 in 2001, rising to 1,170 in 2020. Further,
Figure~\ref{fig:drafts_year} shows the median number of Internet-Drafts
that are posted before an RFC is published. Days to publication and number
of drafts are strongly correlated, suggesting that the time is spent making
changes to the document. This may go some way towards explaining the
decline in output of the IETF: each RFC is taking longer to produce, with
more revisions before publication. 

\begin{figure}
\includegraphics[width=0.45\textwidth]{figures-prev/imc-2021/documents/day_counts_yearly.pdf}
\caption{Days from first draft to RFC publication}
\label{fig:rfcs_days_to_pub}
\end{figure}

\begin{figure}
\includegraphics[width=0.45\textwidth]{figures-prev/imc-2021/documents/draft_counts_yearly.pdf}
\caption{Number of drafts per RFC}
\label{fig:drafts_year}
\end{figure}

%..................................................................................................
\pb{RFC Length:}
One may conjecture that this slowdown is driven by longer RFCs that contain
more material. To explore this, Figure~\ref{fig:pages_year} shows the
median page count of RFCs. This shows that the increase in the duration of
the standardisation process for RFCs cannot be attributed to RFCs becoming
longer: median page counts have remained stable. 

\begin{figure}
\includegraphics[width=0.45\textwidth]{figures-prev/imc-2021/documents/page_counts_yearly_dt.pdf}
\caption{RFC page counts}
\label{fig:pages_year}
\end{figure}

%..................................................................................................
\pb{Relationships between RFCs:}
RFCs themselves may be becoming more complex as a result of them needing to
describe their relationship with older standards. As the Internet has
evolved and matured, applications and protocols must be maintained, and new
standards need to interoperate with existing RFCs.
Figure~\ref{fig:updates_year} highlights this, showing the proportion of
RFCs that are published each year that update (i.e., extend or augment) or
obsolete (i.e., replace) one or more previously published RFCs. Intuitively,
this percentage has slowly increased as the IETF has matured: in 2020, more
than 30\% of RFCs updated or made obsolete a previous RFC.
Figure~\ref{fig:citations_year} expands on this, showing the median number
of citations from each RFC to other Internet-Drafts and RFCs. This
similarly shows that RFCs are increasingly referring to prior work.

\begin{figure}
\includegraphics[width=0.45\textwidth]{figures-prev/imc-2021/documents/update_obsolete_yearly_pct.pdf}
\caption{RFCs that update or obsolete previous RFCs}
\label{fig:updates_year}
\end{figure}

\begin{figure}
\includegraphics[width=0.45\textwidth]{figures-prev/imc-2021/documents/cite_counts_yearly.pdf}
\caption{Citations from RFCs to other Internet-Drafts and RFCs per RFC}
\label{fig:citations_year}
\end{figure}

%..................................................................................................
\pb{Use of requirements-setting language:}
Figure~\ref{fig:keyword_usage_rates} further confirms the growing
complexity of RFCs, showing how the use of keywords has evolved over time.
Keywords are used in RFCs to indicate the normative requirements an RFC
imposes on implementations. Figure~\ref{fig:keyword_usage_rates} shows the
total number of occurrences of each of the RFC 2119~\cite{RFC2119} keywords
(i.e., MUST, MUST NOT, REQUIRED, SHALL, SHALL NOT, SHOULD, SHOULD NOT,
RECOMMENDED, MAY, OPTIONAL), divided by the page count of the RFC. As
shown, the median number of keywords per page grew from 2001 through to
2010, indicating a growing number of requirements being expressed in RFCs,
before plateauing in recent years.

\begin{figure}
\includegraphics[width=0.45\textwidth]{figures-prev/imc-2021/documents/keyword_usage_rate.pdf}
\caption{Keyword occurrences per page}
\label{fig:keyword_usage_rates}
\end{figure}


\todo{Should we include ``Academic impact of RFCs'' from IMC 2021?}


%..................................................................................................
\pb{Summary:}
RFCs are taking longer to produce, and they go through a greater number of
revisions before publication. Further, they increasingly update or
reference previously published RFCs, and make greater use of
requirements-setting keywords. Yet, while these measures indicate that the
overall complexity of standards documents is increasing, none of these
factors strongly correlate with the time taken to publish RFCs. This
indicates that these factors aren't driving the increasing duration of the
standardisation process. In the sections that follow, we explore trends in
authorship and community interaction that may also impact the process.


%--------------------------------------------------------------------------------------------------
\subsection{RFC Errata}

% The following is from our TMA 2023 paper, Section III:

%..................................................................................................
\pb{Errata over Time:}
Figure~\ref{fig:errata_per_year} presents the number of errata filed, on
average per RFC, since 1969 based on the year of RFC publication.  The peak
in the number of errata per RFC occurs in 1981. Only 29 RFCs were published
that year, but they include major documents such as RFCs 791, 792, and 793
(the original versions of the IP~\cite{rfc791}, ICMP~\cite{rfc792}, and
TCP~\cite{rfc793} standards), with 17, 7, and 47 errata, respectively.
These important protocols clearly garnered a great deal of scrutiny and
revision.  The second highest peak occurs in 2006.  In contrast to the
previous examples, this has the highest number of RFCs published per year
(459), including RFC 4601 \cite{RFC4601} that has the most errata (114).
Since this second peak, there has been a steady decrease in the number of
errata filed.  This broadly correlates with the number of RFCs published
per year, with Pearson coefficient 0.59 since 2007.  Table
\ref{tab:top-10-rfcs-by-errata} lists the top RFCs by errata filing count.

\begin{figure}
\includegraphics[width=0.45\textwidth]{figures-prev/tma-2023/errata-by-year.pdf}
\caption{Average errata filed per RFC for each year by RFC publication year,
grouped by IETF area (acronyms expanded in Table~\ref{tab:errata-stats-area}).}
\label{fig:errata_per_year}
\end{figure}

\begin{table*}
\centering
\footnotesize
\begin{tabular}{rlrrr}
\toprule
\textbf{RFC} & \textbf{Title} & \textbf{Year} & \textbf{Area} & \textbf{Filing count} \\
\midrule
4601 & Protocol Independent Multicast - Sparse Mode (PIM-SM): Protocol Specification (Revised) & 2006 & rtg & 114 \\
4880 & OpenPGP Message Format & 2007 & sec & 52 \\
793 & Transmission Control Protocol & 1981 & None & 47 \\
4634 & US Secure Hash Algorithms (SHA and HMAC-SHA) & 2006 & None & 44 \\
5661 & Network File System (NFS) Version 4 Minor Version 1 Protocol & 2010 & tsv & 42 \\
1345 & Character Mnemonics and Character Sets & 1992 & app & 41 \\
8446 & The Transport Layer Security (TLS) Protocol Version 1.3 & 2018 & sec & 40 \\
5545 & Internet Calendaring and Scheduling Core Object Specification (iCalendar) & 2009 & app & 35 \\
3261 & SIP: Session Initiation Protocol & 2002 & rai & 33 \\
5905 & Network Time Protocol Version 4: Protocol and Algorithms Specification & 2010 & int & 32 \\
\bottomrule
\end{tabular}
\caption{Top 10 RFCs by errata filing count}
\label{tab:top-10-rfcs-by-errata}
\end{table*}

\begin{table*}
\centering
\footnotesize
\begin{tabular}{lr|rrrr|rr}
\toprule
\textbf{Area} & \textbf{\#} & \textbf{Verified} & \textbf{Held} & \textbf{Rejected} & \textbf{Reported} & \textbf{Technical} & \textbf{Editorial} \\
\midrule
None & 1883  & 895  & 505  & 197  & 286  & 930  & 953 \\
Internet (int) & 650  & 281  & 223  & 98  & 48  & 342  & 308 \\
Operations and Management (ops) & 558  & 311  & 113  & 67  & 67  & 297  & 261 \\
Real-time Applications and Infrastructure (rai) & 457  & 143  & 213  & 48  & 53  & 255  & 202 \\
Security (sec) & 888  & 291  & 265  & 115  & 217  & 447  & 441 \\
Routing (rtg) & 831  & 305  & 378  & 140  & 8  & 326  & 505 \\
Applications (app) & 787  & 370  & 175  & 116  & 126  & 464  & 323 \\
Transport (tsv) & 459  & 188  & 142  & 75  & 54  & 258  & 201 \\
General (gen) & 41  & 22  & 4  & 5  & 10  & 8  & 33 \\
Applications and Real-Time (art) & 204  & 64  & 32  & 16  & 92  & 143  & 61 \\
Sub-IP (subip) & 1  & 1  & 0  & 0  & 0  & 1  & 0 \\
\midrule All & 6759  & 2871  & 2050  & 877  & 961  & 3471  & 3288 \\
\bottomrule
\end{tabular}
\caption{Errata statistics by area.}
\label{tab:errata-stats-area}
\end{table*}


%..................................................................................................
\pb{Errata Delay:}
We next explore how long it takes for errata to be identified and filed.
Figure~\ref{fig:errata_submission_days} presents a CDF of the number of
days between RFC publication and the errata being filed, broken down based
on IETF area. We see a wide range of delays. 7.3\% of errata are filed
within the first 30 days, suggesting that many RFCs are published with
issues that could have been identified prior to publication.  RFCs from the
General (\emph{gen}) area--describing IETF policies and procedures--have
the longest delay, with a median of 3,458 days, compared to the
Applications and Real-time (\emph{art}) area with a median of 681
days.\footnote{Errata are filed against RFCs within the \emph{subip} area
within a median of 48 days, but this is skewed, with only 19 RFCs being
published in that area.} Editorial errata are typically filed more quickly,
with a median of 987 days, compared to a median of 1,138 days for technical
errata. 


\begin{figure}
\includegraphics[width=0.45\textwidth]{figures-prev/tma-2023/errata-submission-dates-area.pdf}
\caption{Days from RFC publication and errata filing by IETF area (acronyms
  expanded in Table~\ref{tab:errata-stats-area}).}
\label{fig:errata_submission_days}
\end{figure}


%..................................................................................................
\pb{Errata Status:}
Figure~\ref{fig:errata_status} categorises the errata by status and
publication year of the RFC to which they relate. 

\begin{figure}
\includegraphics[width=0.45\textwidth]{figures-prev/tma-2023/errata-by-status.pdf}
\caption{Errata filings by status, by publication year of the RFC.}
\label{fig:errata_status}
\end{figure}

The largest share (42.5\%) of errata are \emph{verified}: errata that have
been has been confirmed as necessary and accurate. This suggests that many
errata are useful to the community. The next largest share (30.3\%) are
those labelled \emph{hold for document update}.  These are errata that are
not a necessary update to the RFC, but may be considered on future
revisions. For example, \emph{erratum 6278} describes an oversight in RFC
8610~\cite{rfc8610}; the solution to this is non-trivial, and so will be
considered in the next version of the specification. Of the 930 RFCs that
have \emph{hold for document update} errata filed against them, only 40\%
have been updated or obsoleted by a subsequent RFC.  We flag that this may
be a cause for concern, or at least a missed opportunity for improvements
to standards.

The third largest category (13\%) is \emph{rejected}, which covers errata
that are invalid (like \emph{erratum 6323}, which was rejected because the
original text was understood to be correct) or proposes a significant
change to the RFC that should be done by publishing a new RFC (like
\emph{erratum 5814}, which was rejected for proposing a significant change,
rather than reporting an error). Such a large fraction of rejected
submissions is unexpected and may flag issues with people's understanding
of the errata process and its place within the wider standardisation
process.  Finally, 14.2\% of errata are \emph{reported} but unverified.
Again, we are surprised to see unverified errata from over a decade ago,
suggesting the process should be expedited. 


%..................................................................................................
\pb{Errata per RFC Area, Status, and Stream:}
Figure~\ref{fig:errata_per_rfc} shows a CDF of the number of errata filed,
per RFC, in each IETF area. Non-IETF RFCs, e.g., IRTF and independent stream
RFCs, and legacy IETF RFCs, are labelled as ``None''.  We confirm errata in
standards are common: of the 4,373 standards-track RFCs in our dataset,
32.7\% have attracted at least one erratum filing.  However, there are
three notable outliers.  First, RFCs published by the Sub-IP (subip) Area
have very few errata, with only 5\% of subip RFCs attracting errata
filings. This is because this temporary area -- established in 2001 and
concluded in 2005 -- only published 19 RFCs, resulting in a far smaller
sample than the other areas. For comparison, the next smallest area,
General (gen), published 39 RFCs. \emph{gen} RFCs attract a greater number
of errata on average, vs. \emph{subip} RFCs, likely due to their broader
relevance.  Second, we see that both the Application (app) and Security
(sec) Areas' RFCs are more likely to have errata filed for them, with
35.9\% of Application and 39\% of Security RFCs attracting at least one
erratum filing.

\begin{figure}
\includegraphics[width=0.45\textwidth]{figures-prev/tma-2023/errata-by-rfc-by-area.pdf}
\caption{CDF of errata filed per RFC by grouped by IETF area (acronyms
  expanded in Table~\ref{tab:errata-stats-area}).}
\label{fig:errata_per_rfc}
\end{figure}

Finally, Table~\ref{tab:errata-stats-area} details the split between
\emph{technical} and \emph{editorial} errata across each area. While there
is broadly an even split, there are areas where one type of errata is more
dominant. For example, in the Routing (rtg) area, 60.8\% of filings are
editorial, while in the Applications (app) area, 59\% were technical.  It
remains to determine why this is the case, and, in particular, to establish
whether there is something inherent about the RFCs published by these areas
that makes them more prone to containing errata, and to containing one type
of errata vs. another. For example, in the Routing area, structured
notation is frequently used to define routing entities; editorial errata
are often filed in those definitions. Targeting such areas with improved
alternate review procedures may be beneficial.

Tables~\ref{tab:errata-stats-stream} and \ref{tab:errata-stats-status}
further categorise errata by the stream and status, at the time of
publication, of each RFC. As expected, the majority of errata are filed
against IETF RFCs and \emph{Proposed Standards} since these make up the
majority of RFCs that are published. However, there are notable differences
in the average number of filings per RFC. \emph{Proposed Standards} (1.01
errata per RFC), \emph{Draft Standards} (1.93), and \emph{Internet
Standards} (2.17) attract a far higher number of errata per RFC than
\emph{Informational} (0.52) or \emph{Experimental} (0.39) documents. This
may be due to the additional readership and attention that standards-track
documents receive, and because they are more likely to be the basis for
future work and protocol extensions.

\begin{table*}
\centering
\footnotesize
\begin{tabular}{lr|rrrr|rr}
\toprule
\textbf{Stream} & \textbf{\#} & \textbf{Verified} & \textbf{Held} & \textbf{Rejected} & \textbf{Reported} & \textbf{Technical} & \textbf{Editorial} \\
\midrule
IETF (6619) & 5797  & 2348  & 1815  & 798  & 836  & 3034  & 2763 \\
IAB (124) & 55  & 25  & 13  & 3  & 14  & 23  & 32 \\
Independent (376) & 330  & 235  & 32  & 28  & 35  & 172  & 158 \\
Legacy (1929) & 510  & 226  & 182  & 39  & 63  & 198  & 312 \\
IRTF (97) & 67  & 37  & 8  & 9  & 13  & 44  & 23 \\
\midrule All & 6759  & 2871  & 2050  & 877  & 961  & 3471  & 3288 \\
\bottomrule
\end{tabular}
\caption{Errata statistics by stream; the ``\emph{Editorial}'' stream has no documents, and is not shown.}
\label{tab:errata-stats-stream}
\end{table*}

\begin{table*}
\centering
\footnotesize
\begin{tabular}{lr|rrrr|rr}
\toprule
\textbf{Status} & \textbf{\#} & \textbf{Verified} & \textbf{Held} & \textbf{Rejected} & \textbf{Reported} & \textbf{Technical} & \textbf{Editorial} \\
\midrule
Proposed Standard (4084) & 4142  & 1680  & 1308  & 555  & 599  & 2213  & 1929 \\
Informational (2894) & 1500  & 719  & 399  & 136  & 246  & 754  & 746 \\
Internet Standard (147) & 319  & 118  & 111  & 66  & 24  & 135  & 184 \\
Best Current Practice (316) & 233  & 111  & 53  & 30  & 39  & 80  & 153 \\
Historic (70) & 20  & 13  & 4  & 2  & 1  & 9  & 11 \\
Draft Standard (142) & 274  & 94  & 93  & 66  & 21  & 150  & 124 \\
Experimental (563) & 221  & 115  & 65  & 20  & 21  & 121  & 100 \\
Unknown (929) & 50  & 21  & 17  & 2  & 10  & 9  & 41 \\
\midrule All & 6759  & 2871  & 2050  & 877  & 961  & 3471  & 3288 \\
\bottomrule
\end{tabular}
\caption{Errata statistics by status at publication.}
\label{tab:errata-stats-status}
\end{table*}


\todo{Should we include ``Impact of Citations'' from TMA 2023?}

%..................................................................................................
\pb{Errata Location:}
Finally, we investigate the location of errata within RFCs.
Figure~\ref{fig:errata_location_percent_count} presents the number of
errata occurring at each decile of the documents, for the 2,552 filings
where accurate location information is available, and after the copyright
notice and other boilerplate has been removed.  We see that technical
errata dominate over editorial in almost all places, except for the very
beginning where the \textit{Introduction} is located. Moreover, it shows
that the most technical errata are near the middle of the document where
the most complex content is.  We explore where errata occur, with
Figure~\ref{fig:errata_section_wise_counts} showing section titles for
errata appearing in at least $10$ documents. Sections such as the
\emph{Introduction} or \textit{References} are dominated by editorial
errata while more technical sections, like \emph{IANA Considerations} (i.e.,
parameter registrations), \emph{Security Considerations} or
\emph{Definitions}, have a larger proportion of technical errata. In
addition, we see that sections labelled \emph{Appendix} attract a
significant proportion of technical errata. While appendices vary in their
content, they are widely used to provide pseudocode and test vectors, or to
describe algorithms. This suggests that it may be useful to target review
efforts on appendices and other dense technical content where errata are
more likely.

\begin{figure}
\includegraphics[width=0.45\textwidth]{figures-prev/tma-2023/location-percent.pdf}
\caption{Errata counts by percentile location in document (0 is the beginning; 100 is the end).}
\label{fig:errata_location_percent_count}
\end{figure}

\begin{figure}
\includegraphics[width=0.45\textwidth]{figures-prev/tma-2023/section-title-counts.pdf}
\caption{Errata counts by section title for the more frequent section titles.}
\label{fig:errata_section_wise_counts}
\end{figure}

%==================================================================================================
\section{Trends in Participant Demographics}

% The following is from our IMC 2021 paper, Section 3.2:


The IETF Datatracker maintains metadata about document authors, including
their names, email addresses, affiliations, and location information. This
dataset does not cover the entire RFC corpus, with metadata available for
authors of RFCs published from 2001, and where it has been provided.
Country data is available for around 70\% of authors, while affiliation
information is given for around 80\%. In this section, we look at the
authorship of all of the RFCs published each year, using the available
metadata. An author is counted once in a year for each affiliation or
location they hold.

%..................................................................................................
\pb{Geographic Distribution of Authors:}
Figures~\ref{fig:author_countries_normalised} and
\ref{fig:author_continents_normalised} show the proportion of authors from
countries and continents, respectively. The IETF has signalled that it
wishes to encourage greater geographical diversity~\cite{rfc7704,
ietfblog:diversity}. Without an explicit goal, we frame our findings within
the context of global population distribution.  We find that North America,
while still disproportionately over represented, is becoming less dominant.
75\% of authors were from North America in 2001, and this has declined to
44\% in 2020. At the same time, representation of both Europe and Asia has
grown, from 17\% to 40\% and 6\% to 14\%, respectively.  However, Africa
and South America remain heavily underrepresented, with only $\approx$0.5\%
of authors coming from either continent in 2020.  This suggests that, if
the IETF is to become more geographically representative, further efforts
are needed.

\begin{figure}
\includegraphics[width=0.45\textwidth]{figures-prev/imc-2021/authors/top5_countries_normalised.pdf}
\caption{Authorship countries (normalised)}
\label{fig:author_countries_normalised}
\end{figure}

\begin{figure}
\includegraphics[width=0.45\textwidth]{figures-prev/imc-2021/authors/continents_normalised.pdf}
\caption{Authorship continents (normalised)}
\label{fig:author_continents_normalised}
\end{figure}

%..................................................................................................
\pb{Author Affiliations:}
Figure~\ref{fig:author_affiliations_normalised} shows the top 10
affiliations by proportion of authors each year. Affiliation data is
gathered from the Datatracker, and processed to normalise affiliation
names, removing common variations in spelling, and amalgamating known
subsidiaries and merged companies. For example, Huawei and Futurewei are
combined as Huawei, and Sun Microsystems is merged with Oracle.

From Figure~\ref{fig:author_affiliations_normalised}, we observe several
interesting trends.  First, Cisco remains a consistent employer of IETF
contributors, with around 12\% of authors affiliated with the company in
2020, and having been the single largest affiliation across all years in
the dataset. We can also see the rise of Huawei beginning in 2005, with
7.1\% of authors being affiliated with the company in 2020, having peaked
at 9.7\% in 2018. Google has a similar trajectory, first appearing in the
dataset in 2006, with 3.8\% of authors being affiliated with it in 2020.

We also observe the decline of a number of affiliations.  Microsoft and
Nokia, having peaked with 3.3\% and 3.6\% of authors, had 0.7\% and 1.7\%
of authors in 2020, respectively, with the absolute number of authors from
both companies also declining.

That we are able to observe changing trends in author affiliations, and in
particular, that participants from new companies contribute, indicates that
the IETF has broader relevance beyond a narrow set of companies. While we
do not demonstrate any causal link, it is encouraging that commercially
successful companies opt to enable their employees to actively participate
in the IETF. Care is needed to ensure that this relevance is maintained,
however: we observe that the author pool has grown less diverse in terms of
companies that are represented. 35.4\% of authors came from the top 10
affiliations in the dataset in 2020, compared with 25.6\% in 2001. 


\begin{figure}
\includegraphics[width=0.45\textwidth]{figures-prev/imc-2021/authors/top5_affiliations_normalised.pdf}
\caption{Authorship affiliations (normalised)}
\label{fig:author_affiliations_normalised}
\end{figure}

%..................................................................................................
\pb{Academia and consultants:}
Academic affiliations are those where the affiliation name contains
``University'', ``Institute'', or ``College'', while consultancy
affiliations are those that contain ``Consultant``. Affiliation data has
been normalised to remove common abbreviations (e.g., ``U.'' for
``University'') and to translate non-English affiliations.  As shown in
Figure~\ref{fig:author_affiliations_normalised}, we find that an increasing
number of authors come from academic affiliations, growing from 8.1\% of
authors in 2001, to 13.6\% in 2020, having peaked at 16.5\% in 2009. The
number of consultants, by our measure, has remained stable, accounting for
2\% of authors in 2020.  The remaining authors are largely from industrial
affiliations.

Figure~\ref{fig:author_affiliations_normalised_acad} shows the top 10
academic affiliations in the dataset, and the percentage of academic
authors that have those affiliations over time. In general, academic
affiliations are each typically held by a small number of authors. We can
see a number of trends in academic authorship, with fewer authors from
Columbia University, MIT, and ISI in recent years, and the rise of Tsinghua
University and University Carlos III of Madrid.

\begin{figure}
\includegraphics[width=0.45\textwidth]{figures-prev/imc-2021/authors/top5_affiliations_normalised_acad.pdf}
\caption{Academic affiliations (normalised)}
\label{fig:author_affiliations_normalised_acad}
\end{figure}

%..................................................................................................
% IMC 2021 paper Section 3.2 (last part)
\pb{Arrival of new authors:}
Finally, Figure~\ref{fig:author_new} shows the percentage of authors in each
year that have not previously authored an RFC. Given that the dataset used here
begins in 2001, 100\% of authors are new in that year. The more stable trend in
recent years likely highlights the churn in RFC authorship, with around 30\%
of authors each year having never previously authored an RFC.
\todo{Not clear this fits here}

\begin{figure}
\includegraphics[width=0.45\textwidth]{figures-prev/imc-2021/authors/repeat_authors.pdf}
\caption{Percentage of new authors per year.}
\label{fig:author_new}
\end{figure}

%..................................................................................................
\pb{Summary:}
The trends highlighted in this section  indicate a pool of authors that is
diversifying and changing over time, with a growing proportion of authors
from outside of North America, contributions from new companies and
affiliations, and relatively high authorship churn. Despite this, we find
that certain regions and groups are not well represented in the IETF (e.g.,
contributors from Africa), and that the authorship pool is becoming
increasingly centralised, with a third of authors coming from the top 10
affiliations. This suggests, if the organisation is concerned with being
more representative, further efforts are needed.

%==================================================================================================
\section{Organisational Dynamics of the IETF}

%--------------------------------------------------------------------------------------------------
\subsection{Characterising Participant Interactions}

% IMC 2021 paper Section 3.3

The operations of the IETF is largely underpinned by mailing list
interactions, which are used to discuss and finalise the drafts that
eventually become RFCs. Our data, which starts from 1995, confirms their
vital role, with 1,153 mailing lists, containing 2,439,240 emails from
74,646 unique email addresses.

\pb{Volume of Discussion:}
Figure~\ref{fig:pid_count_emailing_yearly} presents the number of emails
sent across the last 25 years, showing that email volumes have grown
significantly with time, plateauing at around 130,000 annual messages since
around 2010. The figure also shows how the number of unique Person ID
counts (as described in \S\ref{sec:data}) observed every year in the mail
lists, showing a decreasing trend in the number of contributors.

Figure~\ref{fig:emailvol_by_year_catg} shows the relative email volume by
type. We see that an increasing fraction of messages originate from
automated addresses. This is largely as a result of the growing use of
GitHub, and other version control systems, for managing drafts. There were
122 active IETF working groups at the time of writing: of these, 17 listed
a GitHub repository in their metadata. The QUIC working group, as one
example, has replaced the typical email list discussions with GitHub
issues: indeed, this is a significant part of the surge observed in 2016.
Given that most working groups use mailing lists to manage their activity,
we do not further analyse interactions that take place on GitHub in this
paper. As a result, Figure~\ref{fig:emailvol_by_year_catg} understates the
volume of interactions: the plateau observed in recent years is at least
somewhat attributable to the shift to GitHub and similar services. As this
shift continues, it will become important for future work to consider these
interactions.
\todo{We have more discussion of GitHub?}


\begin{figure}
\includegraphics[width=0.45\textwidth]{figures-prev/imc-2021/emails/pid_count_emailing_yearly.pdf}
\caption{Number of Person IDs exchanging emails each year}
\label{fig:pid_count_emailing_yearly}
\end{figure}


\begin{figure}
\includegraphics[width=0.45\textwidth]{figures-prev/imc-2021/emails/frequency_emails_yearly_categories2.pdf}
\caption{Number of messages exchanged per year across different categories:
  messages mapped to the Datatracker (Datatracker Person-ID); messages by
  automated email addresses (Automated); messages by role-based addresses
  (Role-based); and messages not mapped to the Datatracker (New Person-ID)}
\label{fig:emailvol_by_year_catg}
\end{figure}

%..................................................................................................
\pb{Discussion of drafts:}
The mailing lists are, other than plenary or interim meetings, the primary
means for the discussion of drafts.  To measure how often drafts are
discussed, we next identify mentions of drafts in mailing list messages.
We therefore extract any mention of a draft (beginning \texttt{draft-}) or
RFC (i.e., ``RFC'' followed by a number).  Figure~\ref{fig:draft_mentions}
presents the number of drafts mentions in the emails per year. Separate
mentions of the same draft are counted as different mentions, as we want to
observe the entire volume of mentions.  We observe a strong increase in the
number of mentions over time.  This is largely driven by the growing number
of drafts being published. In fact, we find a Pearson correlation of 0.89
between the number of drafts published and the number of mentions. This
speaks to the influence of emails that mention drafts in driving draft
production.

\begin{figure}
\includegraphics[width=0.45\textwidth]{figures-prev/imc-2021/emails/yearly-draft-mention-volume.pdf}
\caption{Number of draft mentions in each year found in the mail lists}
\label{fig:draft_mentions}
\end{figure}

%..................................................................................................
\pb{Contribution duration.}
We now look into the longevity of contributors, in terms of the number of
years that they actively participate in any IETF mailing lists. We define
the \textit{contribution duration} of a participant as the length of time
that they have contributed to the mailing lists.  To do this, we look at
the contributors who first contribute to mailing lists between the years
2000 to 2013. We limit our analysis to 2013, since the longevity of
contributors that first participate more recently cannot be determined.
For each year, in the period 2000 to 2013, we look at those who first
contribute in each year and the number of years that they then go on to
remain active in the mailing lists. For example, a participant who first
sent an e-mail to an IETF list in 2010, and last sent an e-mail to an IETF
list in 2018, will have a contribution duration of 9 years.

We generate Gaussian Mixture Models for observing different clusters of the
maximum duration that contributors remain active for.  These models
identified three broad clusters that contributors can be categorised
within:

\begin{itemize}
    \item \emph{Young contributors} who leave within 1 year of joining.
    \item \emph{Mid-age contributors} who go onto remain associated for
      more than 1 year, but less than 5 years.
    \item \emph{Senior contributors} who go onto remain associated for 5 or
      more years.
\end{itemize}

We use these contribution categories defined above to characterise the
interactions between contributors belonging to different categories, as
well as the authors of each RFC. We look at outgoing and incoming
interactions with RFC authors in the period between the first draft and
publication of the RFC. If this period is less than two years, we look at
the activities of authors for two years before the RFC was published.
Interactions are defined from the viewpoint of the RFC authors:

\begin{itemize}
    \item \textit{Outgoing interaction}, where an RFC author responds to an
      email from other contributors (i.e., email sent).
    \item \textit{Incoming interaction}, where contributors respond to an
      email by the author (i.e., email received).
\end{itemize}

For each interaction type, we count the number of messages and number of
unique contributors involved in those interactions. For each RFC, we
analyse the interaction activities based on three measures of author
contribution duration:

\begin{itemize}
    \item \emph{Junior-most author}, i.e., the author with the lowest
      contribution duration at the time of publication.
    \item \emph{Senior-most author}, i.e., the author with the highest
      contribution duration at time of publication.
    \item \emph{Mean contribution duration} of all of the authors of the
      RFC at the time of publication.
\end{itemize}

\begin{figure}
\includegraphics[width=0.45\textwidth]{figures-prev/imc-2021/emails/age_authors_RFCs.pdf}
\caption{Contribution duration distribution of authors of RFCs: junior-most
  author of each RFC, senior-most author of each RFC, and mean
  contribution-duration of all authors of each RFC}
\label{fig:age_dist_rfc_authors}
\end{figure}

Figure~\ref{fig:age_dist_rfc_authors} shows the distribution of
contribution duration of each of these three measures. This shows that the
majority of junior-most authors have participated for less than 5 years in
the IETF, whereas the majority of senior-most authors have participated in
the IETF for more than 10 years. In fact, 35\% of authors exceed 15 years
of IETF participation, showing that RFCs tend to be authored by a mix of
seniority levels.

\begin{figure}
\includegraphics[width=0.45\textwidth]{figures-prev/imc-2021/emails/rfc_authors_degree.pdf}
\caption{CDF showing drift in annual degree (interaction with their
  network) of RFC authors over the period 2000-2020.}
\label{fig:degree_dist_rfc_authors}
\end{figure}

%..................................................................................................
\pb{Evolution of interactions:}
Figure~\ref{fig:degree_dist_rfc_authors} shows the gradual drift in annual
degree (i.e., the number of people interacted with) of RFC authors. The
degree of authors has substantially increased over time. For instance, in
the year 2000 only 5.5\% of the authors had a degree of over 25, whereas by
the year 2015 almost a quarter of the authors had a degree over 25.  This
confirms that, on average, more recent RFCs generate greater discussion.
This may explain increasing publication times: authors spend more time
interacting with other participants, and these discussions are likely to
lead to more drafts.
\todo{Replace with material from one of the later papers?}

\begin{figure}
\includegraphics[width=0.45\textwidth]{figures-prev/imc-2021/emails/junior_senior_authors_seniormem_indegree.pdf}
\caption{CDF showing number of senior contributors sending messages
  (in-degree) to junior and senior authors.}
\label{fig:junior_senior_indegree_seniormem}
\end{figure}

We can also contrast the interaction patterns of the junior vs.\ senior
authors. Figure~\ref{fig:junior_senior_indegree_seniormem} presents the
CDFs of the in-degree across both junior and senior authors.  It shows that
the incoming interactions from senior contributors to junior authors are
significantly less than the incoming interactions from senior contributors
to senior authors. Nearly 55\% of junior authors receive messages from
fewer than 10 senior contributors, whereas nearly 65\% of senior authors
receive messages from more than 10 senior contributors.  This speaks to the
differing roles played by these sub-populations: senior authors act as hubs
through which substantial volumes of interaction flow.
\todo{Replace with material from one of the later papers?}

%..................................................................................................
\pb{Summary:}
Emails play a vital role in underpinning RFC publication with approximately
130,000 emails sent per year. We have shown that the seniority of both
participants and RFC authors fundamentally changes the volume of
interactions that they have. These trends are likely to have implications
for the IETF community, especially as it tries to encourage new
participants.



%--------------------------------------------------------------------------------------------------
\subsection{Influence}

% ICWSM 2022 paper
% - Measuring influence
% - Behaviour of influential participants
% - Impact of influential participants


%--------------------------------------------------------------------------------------------------
\subsection{Hierarchy}
%..................................................................................................
% ICWSM 2024 paper
% - RQ1: Are higher levels of the hierarchy growing or becoming more
%   centralised? Are they associated with an increased domination of the
%   conversation? 
% - RQ2: What is the association between the organisation hierarchy and
%   general communication patterns?
% - RQ3: How do people with differing roles communicate with each other?
%   Does information flow “up” or “down” the hierarchy? 
% - RQ4: What is the impact that individuals have on their direct contacts’
%   activity over time? How does this vary across hierarchy level?

%..................................................................................................
% LEDA
% - a taxonomy of dialogue acts (DAs) and a labeled dataset of emails

%==================================================================================================
\section{Understanding Participant Behaviour}

% Tracing Linguistic Markers
% - RQ1: How do linguistic traits differ between more and less influential
%   participants? 
% - RQ2: How do linguistic traits vary for participants at different levels
%   of the organisation hierarchy? 
% - RQ3: How does linguistic behaviour of participants change as they gain
%   influence?

% Frontiers in Psychology paper
% - Use of Language, discussing how it shifts as people become more senior

%==================================================================================================
\section{Success Factors}

\subsection{For authors}
% ICWSM 2022 paper Section 4



\subsection{For documents}

% IMC 2021 paper Section 4

%==================================================================================================
\section{Related Work}

% This should come near the end, and focussing on discussing how your work
% relates to that of others. Any relevant related work should have been
% cited already, so this is not a list of related work, it's a discussion
% of how that work relates.
%
% Why not put related work after the introduction? 1) because describing
% alternative approaches gets between the reader and your idea; and 2)
% because the reader knows nothing about the problem yet, so your
% (carefully trimmed) description of various technical trade-offs is
% absolutely incomprehensible.
%
% When writing the related work:
%  - Give credit to others where it's due; this doesn't diminish the
%    credit you get from your paper.
%  - Acknowledge weaknesses in your approach.
%  - Ensure related work is accurate and up-to-date



%==================================================================================================
\section{Conclusions}



%==================================================================================================
\section*{Acknowledgements}

% Acknowledge funding sources.

%==================================================================================================
\bibliographystyle{abbrv}
\bibliography{paper}

%==================================================================================================
% The following information gets written into the PDF file information:
\ifpdf
  \pdfinfo{
    /Title        (...)
    /Author       (...)
    /Subject      (...)
    /Keywords     (..., ..., ...)
    /CreationDate (D:20150827110616Z)
    /ModDate      (D:20150827110616Z)
    /Creator      (LaTeX)
    /Producer     (pdfTeX)
  }
  % Suppress unnecessary metadata, to ensure the PDF generated by pdflatex is
  % identical each time it is built. This needs pdfTeX 3.14159265-2.6-1.40.17
  % or later.
  \ifdefined\pdftrailerid
    \pdftrailerid{}
    \pdfsuppressptexinfo=15
  \fi
\fi
%==================================================================================================
\end{document}
% vim: set ts=2 sw=2 tw=75 et ai:
