\documentclass[twocolumn,10pt]{article}
\usepackage[l2tabu,orthodox]{nag}
\usepackage[utf8x]{inputenc}
\usepackage[british]{babel}
\usepackage{microtype}
\usepackage{amsmath}
\usepackage[all]{onlyamsmath}
\usepackage{newtxtext}
\usepackage{newtxmath}
\usepackage[caption=false]{subfig}
\usepackage{booktabs}
\usepackage{upquote}
\usepackage{graphicx}
\usepackage{url}
\usepackage{algorithm}
\usepackage{algpseudocode}
\usepackage{color}
\usepackage[cm]{fullpage}

\frenchspacing
\uchyph=0

\newlength{\figureWidthOneColumn}
\setlength{\figureWidthOneColumn}{0.4\textwidth}

\newlength{\figureWidthTwoColumn}
\setlength{\figureWidthTwoColumn}{0.8\textwidth}

\newcommand{\todo}[1]{\textbf{\textcolor{red}{To do: #1}}}
\newcommand{\pb}[1]{\vspace{0.75ex}\noindent{\textbf{#1}}}

%==================================================================================================
\begin{document}

\title{An Empirical Analysis of the Internet Engineering Task Force 
       with Computational Methods}
\author{Mladen Karan \and
        Prashant Khare \and
        Matthew Russell Barnes \and
        Waleed Iqbal \and
        Junaid Qadir \and
        Stephen McQuistin \and
        Richard G. Clegg \and
        Patrick Healey \and
        Matthew Purver \and
        Ignacio Castro \and
        Gareth Tyson \and
        Colin Perkins}

\maketitle
%==================================================================================================
\begin{abstract}
  % Four sentences:
  %  - State the problem
  %  - Say why it's an interesting problem
  %  - Say what your solution achieves
  %  - Say what follows from your solution

  The Internet requires interoperability between networks, systems, and
  applications, as well as cooperation among a growing number of
  stakeholders. The Internet Engineering Task Force (IETF) is critical in
  supporting this cooperation and interoperability by bringing together
  interested parties and standardising the protocols that power the
  Internet, such as IPv4/6 or HTTP/s. Our work covers more 20 years of IETF
  data, and analyses the participants of the IETF, the emails they
  exchange, and the documents they produce. We show how these data can be
  used to understand who are these Internet actors, and their interests as
  well as how Internet's growth and maturity has given rise to a longer and
  complex standardisation process where participants gain and exercise
  influence and how this is reflected in the language they use and the
  interactions they have.
  

\end{abstract}
%==================================================================================================
\section{Introduction}

% A good paper introduction is fairly formulaic. If you follow a simple set
% of rules, you can write a very good introduction. The following outline can
% be varied. For example, you can use two paragraphs instead of one, or you
% can place more emphasis on one aspect of the intro than another. But in all
% cases, all of the points below need to be covered in an introduction, and
% in most papers, you don't need to cover anything more in an introduction.
%
% Paragraph 1: Motivation. At a high level, what is the problem area you
% are working in and why is it important? It is important to set the larger
% context here. Why is the problem of interest and importance to the larger
% community?



% Paragraph 2: What is the specific problem considered in this paper? This
% paragraph narrows down the topic area of the paper. In the first
% paragraph you have established general context and importance. Here you
% establish specific context and background.



% Paragraph 3: "In this paper, we show that...". This is the key paragraph
% in the introduction - you summarize, in one paragraph, what are the main
% contributions of your paper, given the context established in paragraphs
% 1 and 2. What's the general approach taken? Why are the specific results
% significant? The story is not what you did, but rather:
%  - what you show, new ideas, new insights
%  - why interesting, important?
% State your contributions: these drive the entire paper.  Contributions
% should be refutable claims, not vague generic statements.

In this paper, we ...

% Paragraph 4: What are the differences between your work, and what others
% have done? Keep this at a high level, as you can refer to future sections
% where specific details and differences will be given, but it is important
% for the reader to know what is new about this work compared to other work
% in the area.


This paper summarises and expands on our prior work
\cite{mcquistin:2021:characterising,khare:2022:web-we-weave,
healey:2023:power,mcquistin:2023:errare,khare:2023:tracing,
karan:2023:leda,healey:2023:power-frontiers,barnes:2024:temporal}.

% Paragraph 5: "We structure the remainder of this paper as follows." Give
% the reader a road-map for the rest of the paper. Try to avoid redundant
% phrasing, "In Section 2, In section 3, ..., In Section 4, ... ", etc.

We structure the remainder of this paper as follows.
Section \ref{sec:background} introduces the IETF and the Internet Standards
development process, and reviews the data with which we work.
Section \ref{sec:trends-documents} considers trends in document production,
complexity, and correctness.
Section \ref{sec:trends-demographics} moves on to consider the people who
are involved in Internet standards development and how the demographics of
the community have shifted over time.
Section \ref{sec:org-dyn} studies the interaction between participants,
explores who has influence and impact in the IETF and how that influence
is evident in their activities, and the dynamics of communications within
the IETF.
Section \ref{sec:language} considers communications within the IETF from
the perspective of language to explore whether there are differences in
linguistic patterns and language use in more influential participants,
and to consider how the use of language reflects the consensus-driven
nature of the standards development process.
Section \ref{sec:success-factors} then considers factors that lead to
success, exploring what makes a document likely to be completed and
used, and what makes a successful participant.
Finally, we review related work in Section \ref{sec:related} and conclude
in Section \ref{sec:conclusions}.

%==================================================================================================
\section{Background and Datasets}
\label{sec:background}

% The following is from our IMC 2021 paper:

We start by presenting an overview of the IETF, and the publication process
for RFCs, before outlining the data sources we use within this paper. We
also highlight the ethical considerations of accessing and processing this
data.


%--------------------------------------------------------------------------------------------------
\subsection{An IETF Primer}
\label{sec:background:ietf}

% The following is from our IMC 2021 paper:

%..................................................................................................
\pb{The IETF} is an open standards organisation, which develops
Internet standards via contributions and collaborations across a number of
voluntary stakeholders, including academics, consultants, industry
representatives, governments, and civil society organisations.  Through
extensive collaboration across contributors, the IETF, and associated
organisations, develops Internet standards and other documents. These are
published by the RFC Editor (\url{https://www.rfc-editor.org}) in four
\emph{publication streams}: the IETF stream, the Internet Research Task
Force (IRTF) stream, the Internet Architecture Board (IAB) stream, and the
Independent Submission stream.

There is also a fifth, legacy stream, comprising RFCs published prior to
the adoption of separate publication streams in July 2007 \cite{RFC4844}.
While the IETF is an open standards forum that develops technical standards
and operational guidelines for the Internet, the IRTF is an associated
organisation that promotes longer-term research, and the IAB provides
long-range technical direction for Internet development. The Independent
Submission stream ``allows RFC publication for some documents that are
outside the official IETF/IAB/IRTF process but are relevant to the Internet
community'' (\url{https://www.rfc-editor.org/about/independent}).

%..................................................................................................
\pb{The standards development process} is an inherently collaborative
activity.  Most day-to-day work is conducted on public mailing lists, in
conjunction with three plenary meetings and numerous interim working group
meetings per year. 
The mailing lists are broadly split into three categories: announcement
lists, where replies are not allowed; non-working group lists, for
discussing topics that do not relate to the work within an IETF working
group or IRTF research group; and working group and area lists, where
technical discussions take place.

The process of RFC publication begins with the submission of an
\emph{Internet-Draft}. Whereas anyone can post a draft, not all drafts
become RFCs. After a draft is first posted, multiple revisions might take
place resulting in multiple versions of the draft. Each new draft is
announced on one or more mailing lists related to the topic of the draft,
soliciting feedback and encouraging discussion. Drafts are initially posted
by individuals. For publication under the IETF stream, drafts must then be
adopted by a working group, where, via further revision, the draft may
ultimately be published as an RFC. The process of managing drafts, and the
degree and type of peer review conducted prior to their submission to the
RFC Editor for publication, differs between streams.

Once the technical development of the draft is complete its publication is
managed by the RFC Editor, who maintains the master archive of the RFC
documents, along with an index of metadata pertaining to the RFCs and their
authors. Finally, once an RFC has been published, deployment is voluntary,
and therefore not all RFCs are widely implemented.

%--------------------------------------------------------------------------------------------------
\subsection{Data Sources}
\label{sec:background:data}

% The following is adapted from our IMC 2021 paper and our IFIP TMA 2023
% paper, Section II:

We rely on a number of data sources to conduct our study.

%..................................................................................................
\pb{IETF Datatracker:}
The Datatracker\footnote{\url{https://datatracker.ietf.org/}} is the main
administrative database used by the IETF, IRTF, and IAB to coordinate their
work. It contains information about contributors, working groups,
documents, and meetings. The Datatracker provides comprehensive metadata
about authors, and the evolution of drafts as they work their way through
the standardisation process. The Datatracker was introduced in the early
2000s and its functionality has gradually been extended over time. It
contains limited historical data about RFCs produced before its creation,
but the IETF has back-filled some data from other records.  We have
extracted relevant metadata from the Datatracker, using its programmatic
REST API, for all RFCs published since 2001. This gives us data pertaining
to 4,512 authors, as well as richer metadata for 5,707 RFCs.

%..................................................................................................
\pb{RFC Editor:}
The RFC Editor maintains an index of all RFC publications, alongside
metadata related to each document, including its standardisation status,
publication stream, authors, and errata. We gather all entries for RFCs
published through the end of 2020, giving a total of 8,711 RFCs.

RFCs are not changed after publication, but the RFC Editor maintains a
public database of reported errata. The errata filing are separated into
\emph{technical errata}, mistakes in the technical content of the RFC that
are likely to result in incorrect, non-conforming, implementations of the
standards, and \emph{editorial errata} including spelling and punctuation
errors that do not otherwise impact the technical content.
Errata filed against RFCs published on the IETF stream are checked for
correctness by the RFC authors, working group chairs, and area directors.
IRTF errata are checked by the authors and the Internet Research Steering
Group. The IAB and Independent Submissions Editor check errata for their
streams. We collected\footnote{RFC errata are available at
\url{https://www.rfc-editor.org/errata.php}. We thank the RFC Editor for
making the underlying database available to us in machine-readable form for
this analysis.} the 6,759 errata reports submitted from January 2001 to the
end of December 2022 for analysis.

%..................................................................................................
\pb{Email archives and entity resolution:}
The IETF maintains reasonably complete email archives relating to working
group discussions, meeting, and other activities.\footnote{\url{https://mailarchive.ietf.org/}
with public IMAP access also available.} We gather all available messages
contained within this archive, covering 2,439,240 messages, sent from
74,646 unique email addresses, across 1,153 lists. This snapshot was taken
on April 18th 2021.

A key challenge is attributing each email to an individual contributor.
Although each, naturally, contains a \texttt{From:} header, we find that some
contributors use multiple addresses. Beyond this, it is necessary to map
each email to the respective person in the Datatracker and RFC Editor
datasets.  Thus, we perform entity resolution on email senders, and match
email senders with their Datatracker profiles. We assign each sender a
unique identifier, which is associated with a set of name and address
variations from the Datatracker. 

Entity resolution takes place in multiple stages.  First, we check if the
sender's email address has a Datatracker profile. If so, we associate all
of their Datatracker metadata with that person's unique identifier, and
label the message as having been sent by that identifier.  Next, if an
email's sender does not appear in the Datatracker, we check, based on their
name, if they have already been assigned a Datatracker record. If so, the
message is labelled as having been sent by that identifier, and the set of
names and addresses associated with that identifier is updated to include
the email's sender name and address.

These two stages -- matching with the Datatracker, and merging previously
seen names and addresses -- accounts for the majority (60\%) of messages.
If an email's sender is not in the Datatracker, and the name and address
has not been previously seen, a new person identifier is generated. This
accounts for a small portion (~10\%) of messages. This is reasonable, given
that Datatracker profiles have become necessary for many of the IETF's
day-to-day activities.

As a final processing step, we label each identifier as either representing
a \emph{contributor}, a \emph{role-based} address, or an \emph{automated}
sender. A contributor refers to any standard user participating in the
IETF; role-based identifiers reflect addresses that are used by the holders
of a particular organisational role, such as the IETF chair; and automated
identifiers are those addresses that are system-specific, such as GitHub or
IETF notification and announcement addresses. Role-based and automated
identifiers account for the remaining 30\% of messages.

Note, within the mail archive, spam-indicating headers are present for most
messages since 2009, and we confirm pre-filtering is performed by the IETF
mail servers.  As an extra validation step, we ran a spam filter over all
the archived messages and discarded those marked as spam. Both sources
indicate there is very little spam (less than 1\%), so it should not have
significant impact on our findings.

% \pb{Manually labelled dataset:} 
% We also use data collated by Nikkhah et al.~\cite{nikkhah2017statistical}.
% This dataset labels the ``success'', alongside other features like scope
% and value, of 251 RFCs.  155 of these also appear in our set of RFCs that
% have Datatracker metadata available. We later use this to explore the
% features that best predict the success of an RFC.

% \pb{Microsoft Academic citation data:}
% We use the Microsoft Academic Knowledge API to identify indexed academic
% work that cites each RFC. We use the Microsoft Academic Graph instead of
% other sources (\eg Google Scholar) as it provides time-stamped citations.

%..................................................................................................
\pb{Reproducibility and data access:}
We have developed a library, \texttt{ietfdata}, that fetches the RFC
Index, communicates with the IETF Datatracker API, and retrieves messages
from the IETF IMAP mail archives. This library appropriately regulates
access, caches data to minimise the impact on the infrastructure, and
performs necessary post-processing. This library is available as open
source and Ongoing development is coordinated on GitHub.%
\footnote{\url{https://github.com/glasgow-ipl/ietfdata}}


%--------------------------------------------------------------------------------------------------
\subsection{Ethical Considerations}
\label{sec:ethics}

% The following is adapted from our IMC 2021 paper:

The data we analyse is extracted from public IETF archives and APIs.  We
have taken steps to ensure ethical compliance.  To ensure that our access
to these services does not cause operational problems for the IETF, we are
in regular contact with the IETF Tools Team and Secretariat, as well as the
operators of the Datatracker and mailing list archive.  We have
discussed our work with IETF leadership (IETF and IAB Chairs, the IETF
Executive Director, and the IRTF Chair, who is a co-author on this paper)
and the RFC Editor to ensure that our access falls within their acceptable
use policies.

Participants in the IETF must agree to abide by the policies and procedures
described at \url{https://www.ietf.org/about/note-well}, including the privacy
policy at \url{https://www.ietf.org/privacy-statement}. These make explicit
provision that mailing list archives and the metadata contained in the
Datatracker system will be made public, and it is this public data that
we process to extract the aggregate statistics presented in Sections
\ref{sec:trends-documents}, \ref{sec:trends-demographics}, \ref{sec:org-dyn},
\ref{sec:language} as well as the features analysed in Section
\ref{sec:success-factors}. We transfer and store data securely, and retain
it only for the time needed to perform the analysis.  Since we operate
entirely using the public APIs provided by IETF, we have no access to
private data about individuals.

In balancing these ethical considerations with the reproducibility of our
work, we provide the tools needed to access the datasets from the relevant
IETF sources, rather than the data itself.


%==================================================================================================
\section{Trends in RFC Publication}
\label{sec:trends-documents}

% \begin{figure}
%   \centering
%   \includegraphics{figures/rfcs-by-year-stream.pdf}
%   \caption{Number of RFCs published per year}
%   \label{fig:rfcs-by-year}
% \end{figure}

To begin, we review trends in IETF standards publication. We consider the
rate of publication of documents in the RFC series, factors affecting the
publication rate, and the occurrence of errata reports, and answer the
following research questions:
\begin{enumerate}
  \item What factors affect the publication rate of RFCs and is the IETF
    standards development process getting faster or slower over time?
    (\S\ref{sec:trends-documents:production})

  \item Are the standards produced by the IETF broadly correct? Do the
    reported errata indicate problems with the standards development
    process in general, or in particular areas, and is the IETF getting
    better or worse at producing correct RFCs over time?
    (\S\ref{sec:trends-documents:errata})

\end{enumerate}

%--------------------------------------------------------------------------------------------------
\subsection{RFC Production and Complexity}
\label{sec:trends-documents:production}

RFC production trends show clear evidence that the IETF is a mature
standards development organisation. The peak of activity was in the 1990s
and early 2000s, with the initial broad deployment of the Internet, and the
organisation has since entered more of a development and maintenance mode
of operation. There are clear signs that the complexity of the installed
base of protocols is slowing development of new technologies.

% The following is from our IMC 2021 paper, Section 3.1:

%..................................................................................................
\pb{RFC Publication Rate:}
In total, 8,711 RFCs have been published through to the end of 2020.
Figure~\ref{fig:rfcs_by_area} shows how publication trends, in terms of
IETF areas and non-IETF streams, have changed over time. We identify three
broad publication phases in the RFC series. First, in 1969 through 1974,
RFCs are published at a rapid rate during the initial development of the
ARPANET. Then, from 1975 through 1985, development slows. This reflects a
relatively small community that is gaining real-world experience with the
network and developing a small core of applications and protocols. Finally,
with the creation of the IETF and the introduction of the National Science
Foundation Network (NSFNET) in 1986, both the community, and the number of
RFCs published, starts to expand rapidly. This is further driven by the
opening of the network to commercial and public use in the mid-1990s.
The annual RFC publication rate was highest in 2005, at the peak
of the standardisation efforts for SIP and related standards for voice-over-IP
and Internet telephony. The rate of publication has slowed in recent years,
following the completion of large work programmes relating to HTTP/2 and
WebRTC.

\begin{figure}
  \centering
  \includegraphics[width=\figureWidthOneColumn]{figures-prev/imc-2021/documents/rfcs_areas.pdf}
  \caption{
    Number of RFCs published over time, subdivided by IETF Area.
    The areas are 
    \emph{art} -- applications and real-time;
    \emph{app} -- applications;
    \emph{int} -- internetworking;
    \emph{ops} -- operations and management;
    \emph{rai} -- real-time applications and infrastructure;
    \emph{rtg} -- routing;
    \emph{sec} -- security; and
    \emph{tsv} -- transport and services.
    The \emph{other} category includes legacy RFCs, pre-dating the creation
    of the IETF, and RFCs published in non-IETF publication streams such as
    the IRTF, IAB, and the Independent stream.
  }
  \label{fig:rfcs_by_area}
\end{figure}

\todo{Plot RFC publication count as a cumulative distribution}

In addition to the published RFCs, we track work-in-progress documents,
known as Internet-drafts, that are intended for eventual publication as
RFCs.
\todo{Add internet-draft counts over time as mentioned in Section \ref{sec:org-dyn:participants}}

%..................................................................................................
\pb{Role of Working Groups:}
From the creation of the IETF in 1986 the growing community, with its
interests in an increasing set of applications and protocols, has been
split into a number of \emph{working groups}.\footnote{There are 122
active working groups at the time of writing, organised into 7 areas.
The number of groups varies over time, but is typically in the range
100-150.} These working groups are each chartered with a well-defined
programme of work and exist within \emph{areas} that have a broader focus.  

As shown in Figure~\ref{fig:rfcs_by_area}, the output of
different areas has remained relatively stable over time. The most notable
trends begin with the creation of the Real-time Applications and
Infrastructure (rai) area from within the Transport (tsv) area, and its
later merger with the Applications (app) area to become the Applications
and Real-Time (art) area around 2014. Additionally, we also observe the
significant growth in output of the Routing (rtg) area, owing to the
ongoing development of standards for MPLS, service function chaining, and
fat tree routing in data centres.

To give a sense for the broader productivity of the IETF,
Figure~\ref{fig:pub_wgs_yearly} shows the number of working groups that
publish RFCs each year. This highlights how the structure of the IETF has
grown to accommodate its larger community: in the early 1990s, fewer than
20 working groups were published an RFC each year, while in recent years
there has typically been at least 60 working group publishing RFCs, with
a peak of 97 working groups publishing RFCs in 2011 (the number of groups
publishing RFCs is less than the total number of groups, because it takes
time for the work of a new group to reach maturity).

\begin{figure}
  \centering
  \includegraphics[width=\figureWidthOneColumn]{figures-prev/imc-2021/documents/unique_wgs_per_year_areas.pdf}
  \caption{
    Number of groups publishing RFCs over time, subdivided by IETF Area.
    The areas are as described in Figure~\ref{fig:rfcs_by_area}.
  }
  \label{fig:pub_wgs_yearly}
\end{figure}

%..................................................................................................
\pb{Factors Affecting Publication Rate:} 
The standards development process is taking longer.
Figure~\ref{fig:rfcs_days_to_pub} plots the median number of days from the
submission of the first draft of a document through to its publication as
an RFC. This shows a clear trend: RFCs are taking longer to make their way
through the standardisation and publication process. The median number of
days to publication was 469 in 2001, rising to 1,170 in 2020. 

\begin{figure}
  \centering
  \includegraphics[width=\figureWidthOneColumn]{figures-prev/imc-2021/documents/day_counts_yearly.pdf}
  \caption{
    Days from first draft to RFC publication
  }
  \label{fig:rfcs_days_to_pub}
\end{figure}

Similarly, Figure~\ref{fig:drafts_year} shows the median number of
revisions each document undergoes before being published as an RFC. Days to
publication and number of revisions are strongly correlated, suggesting
that the time is spent making changes to the document. This may go some way
towards explaining the decline in output of the IETF: each RFC is taking
longer to produce, with more revisions before publication. 

\begin{figure}
  \centering
  \includegraphics[width=\figureWidthOneColumn]{figures-prev/imc-2021/documents/draft_counts_yearly.pdf}
  \caption{
    Number of drafts per RFC
  }
  \label{fig:drafts_year}
\end{figure}


What factors affect the publication rate?  One may conjecture that this
slowdown is driven by longer RFCs that contain more material. Yet from the
median page count of RFCs, shown in Figure~\ref{fig:pages_year}, no clear
trend being visible. The increase in the duration of the standardisation
process for RFCs cannot be attributed to RFCs becoming longer: page counts
have remained stable. 

\begin{figure}
  \centering
  \includegraphics[width=\figureWidthOneColumn]{figures-prev/imc-2021/documents/page_counts_yearly_dt.pdf}
  \caption{
    RFC page counts
  }
  \label{fig:pages_year}
\end{figure}

The publication rate might also be slowing because RFCs are becoming more
complex. This could occur, for example, as a result of having to maintain
compatibility with older RFCs as the Internet has evolved and matured.
Figure~\ref{fig:updates_year} shows that this is the case, plotting the
proportion of RFCs that are published each year that update (i.e., extend
or augment) or obsolete (i.e., replace) one or more previously published
RFCs.  This percentage has slowly increased as the IETF has matured: in
2020, more than 30\% of RFCs updated or made obsolete a previous RFC.
%
Figure~\ref{fig:citations_year} expands on this, showing the median number
of citations from each RFC to other Internet-Drafts and RFCs. This
similarly shows that RFCs are increasingly referring to prior work.

\begin{figure}
  \centering
  \includegraphics[width=\figureWidthOneColumn]{figures-prev/imc-2021/documents/update_obsolete_yearly_pct.pdf}
  \caption{
    RFCs that update or obsolete previous RFCs
  }
  \label{fig:updates_year}
\end{figure}

\begin{figure}
  \centering
  \includegraphics[width=\figureWidthOneColumn]{figures-prev/imc-2021/documents/cite_counts_yearly.pdf}
  \caption{
    Number of citations to other Internet-Drafts and RFCs per RFC in each
    year.
  }
  \label{fig:citations_year}
\end{figure}

Figure~\ref{fig:keyword_usage_rates} further confirms the growing complexity
of RFCs, showing how the use of normative language keywords has evolved
over time.  Keywords are used in RFCs to indicate the normative requirements
an RFC imposes on implementations. Figure~\ref{fig:keyword_usage_rates}
shows the total number of occurrences of each of the RFC 2119~\cite{RFC2119}
keywords (i.e., MUST, MUST NOT, REQUIRED, SHALL, SHALL NOT, SHOULD, SHOULD
NOT, RECOMMENDED, MAY, OPTIONAL), divided by the page count of the RFC. As
shown, the median number of keywords per page grew from 2001 through to
2010, indicating a growing number of requirements being expressed in RFCs,
before plateauing in recent years.

\begin{figure}
  \centering
  \includegraphics[width=\figureWidthOneColumn]{figures-prev/imc-2021/documents/keyword_usage_rate.pdf}
  \caption{
    Number of occurrences of normative language keywords per page in RFCs
    in each year.
  }
  \label{fig:keyword_usage_rates}
\end{figure}

%..................................................................................................
\pb{Summary:}
RFCs are taking longer to produce, and they go through a greater number
of revisions before publication. Further, they increasingly update or
reference previously published RFCs, and make greater use of
requirements-setting keywords. These are all hallmarks of a maturing
standards development organisation that must increasingly consider
compatibility with the installed based in the development of its standards.


%--------------------------------------------------------------------------------------------------
\subsection{RFC Errata}
\label{sec:trends-documents:errata}

% The following is from our TMA 2023 paper, Section I:

Mistakes can and do occur in published RFCs.  These errors can be reported
to the RFC Editor, and the RFC Editor makes the reports publicly available
and coordinates with the IETF, and the other publication streams, on their
verification. These errata can clarify editorial concerns as well as correct
substantive technical errors. Since the presence of significant errors in
its specification can undermine the success of a protocol, studying the
nature of these errata and their subsequent fixes is important.

In the following, we analyse the 6,759 errata filed with the RFC Editor
between 2001–2022, inclusive, documenting 3,288 editorial issues and 3,471
technical issues, and covering 2,240 RFCs, to understand the frequency
and causes of problems in RFCs.

% The following is from our TMA 2023 paper, Section III:

%..................................................................................................
\pb{Errata over Time:}
Figure~\ref{fig:errata_per_year} presents the number of errata filed, on
average per RFC, since 1969 based on the year of RFC publication.  The peak
in the number of errata per RFC occurs in 1981. Only 29 RFCs were published
that year, but they include major documents such as RFCs 791, 792, and 793
(the original versions of the IP~\cite{RFC791}, ICMP~\cite{RFC792}, and
TCP~\cite{RFC793} standards), with 17, 7, and 47 errata, respectively.
These important protocols clearly garnered a great deal of scrutiny and
revision.  The second highest peak occurs in 2006.  In contrast to the
previous examples, this has the highest number of RFCs published per year
(459), including RFC 4601 \cite{RFC4601} that has the most errata (114).
Since this second peak, there has been a steady decrease in the number of
errata filed.  This broadly correlates with the number of RFCs published
per year, with Pearson coefficient 0.59 since 2007.
Table \ref{tab:top-10-rfcs-by-errata} lists the top RFCs by errata filing
count.

\begin{figure}
  \centering
  \includegraphics[width=\figureWidthOneColumn]{figures-prev/tma-2023/errata-by-year.pdf}
  \caption{
    Average number of errata filed per RFC in each year, by RFC publication year,
    grouped by IETF area. The areas are as described in Figure~\ref{fig:rfcs_by_area}.
  }
  \label{fig:errata_per_year}
\end{figure}

\begin{table*}
\centering
\footnotesize
\begin{tabular}{rlrrr}
\toprule
\textbf{RFC} & \textbf{Title} & \textbf{Year} & \textbf{Area} & \textbf{Filing count} \\
\midrule
4601 & Protocol Independent Multicast - Sparse Mode (PIM-SM): Protocol Specification (Revised) & 2006 & rtg & 114 \\
4880 & OpenPGP Message Format & 2007 & sec & 52 \\
793 & Transmission Control Protocol & 1981 & None & 47 \\
4634 & US Secure Hash Algorithms (SHA and HMAC-SHA) & 2006 & None & 44 \\
5661 & Network File System (NFS) Version 4 Minor Version 1 Protocol & 2010 & tsv & 42 \\
1345 & Character Mnemonics and Character Sets & 1992 & app & 41 \\
8446 & The Transport Layer Security (TLS) Protocol Version 1.3 & 2018 & sec & 40 \\
5545 & Internet Calendaring and Scheduling Core Object Specification (iCalendar) & 2009 & app & 35 \\
3261 & SIP: Session Initiation Protocol & 2002 & rai & 33 \\
5905 & Network Time Protocol Version 4: Protocol and Algorithms Specification & 2010 & int & 32 \\
\bottomrule
\end{tabular}
\caption{Top 10 RFCs by errata filing count}
\label{tab:top-10-rfcs-by-errata}
\end{table*}

\begin{table*}
\centering
\footnotesize
\begin{tabular}{lr|rrrr|rr}
\toprule
\textbf{Area} & \textbf{\#} & \textbf{Verified} & \textbf{Held} & \textbf{Rejected} & \textbf{Reported} & \textbf{Technical} & \textbf{Editorial} \\
\midrule
None & 1883  & 895  & 505  & 197  & 286  & 930  & 953 \\
Internet (int) & 650  & 281  & 223  & 98  & 48  & 342  & 308 \\
Operations and Management (ops) & 558  & 311  & 113  & 67  & 67  & 297  & 261 \\
Real-time Applications and Infrastructure (rai) & 457  & 143  & 213  & 48  & 53  & 255  & 202 \\
Security (sec) & 888  & 291  & 265  & 115  & 217  & 447  & 441 \\
Routing (rtg) & 831  & 305  & 378  & 140  & 8  & 326  & 505 \\
Applications (app) & 787  & 370  & 175  & 116  & 126  & 464  & 323 \\
Transport (tsv) & 459  & 188  & 142  & 75  & 54  & 258  & 201 \\
General (gen) & 41  & 22  & 4  & 5  & 10  & 8  & 33 \\
Applications and Real-Time (art) & 204  & 64  & 32  & 16  & 92  & 143  & 61 \\
Sub-IP (subip) & 1  & 1  & 0  & 0  & 0  & 1  & 0 \\
\midrule All & 6759  & 2871  & 2050  & 877  & 961  & 3471  & 3288 \\
\bottomrule
\end{tabular}
\caption{Errata statistics by area.}
\label{tab:errata-stats-area}
\end{table*}


%..................................................................................................
\pb{Errata Delay:}
We next explore how long it takes for errata to be identified and filed.
Figure~\ref{fig:errata_submission_days} presents a CDF of the number of
days between RFC publication and the errata being filed, broken down based
on IETF area. We see a wide range of delays. 7.3\% of errata are filed
within the first 30 days, suggesting that many RFCs are published with
issues that could have been identified prior to publication.  RFCs from the
General (\emph{gen}) area--describing IETF policies and procedures--have
the longest delay, with a median of 3,458 days, compared to the
Applications and Real-time (\emph{art}) area with a median of 681
days.\footnote{Errata are filed against RFCs within the \emph{subip} area
within a median of 48 days, but this is skewed, with only 19 RFCs being
published in that area.} Editorial errata are typically filed more quickly,
with a median of 987 days, compared to a median of 1,138 days for technical
errata. 


\begin{figure}
  \centering
  \includegraphics[width=\figureWidthOneColumn]{figures-prev/tma-2023/errata-submission-dates-area.pdf}
  \caption{
    Cumulative distribution of the number of days from RFC publication to
    errata filing by IETF area.
    The areas are as described in Figure~\ref{fig:rfcs_by_area}.
  }
  \label{fig:errata_submission_days}
\end{figure}


%..................................................................................................
\pb{Errata Status:}
Figure~\ref{fig:errata_status} categorises the errata by verification
status and publication year of the RFC to which they relate. 
The largest share (42.5\%) of errata are \emph{verified}: errata that have
been has been confirmed as necessary and accurate. This suggests that many
errata identify real problems with RFCs and are useful to the community.


\begin{figure}
  \centering
  \includegraphics[width=\figureWidthOneColumn]{figures-prev/tma-2023/errata-by-status.pdf}
  \caption{
    Errata filings by status, by publication year of the RFC.
  }
  \label{fig:errata_status}
\end{figure}

The next largest share (30.3\%) are those labelled \emph{hold for document
update}.  These are errata that are not a necessary update to the RFC, but
may be considered on future revisions. For example, \emph{erratum 6278}
describes an oversight in RFC 8610~\cite{RFC8610}; the solution to this is
non-trivial, and so will be considered in the next version of the
specification. Of the 930 RFCs that have \emph{hold for document update}
errata filed against them, only 40\% have been updated or obsoleted by a
subsequent RFC.  We flag that this may be a cause for concern, or at least
a missed opportunity for improvements to standards.

The third largest category (13\%) is \emph{rejected}, which covers errata
that are invalid (like \emph{erratum 6323}, which was rejected because the
original text was understood to be correct) or proposes a significant
change to the RFC that should be done by publishing a new RFC (like
\emph{erratum 5814}, which was rejected for proposing a significant change,
rather than reporting an error). Such a large fraction of rejected
submissions is unexpected and may flag issues with people's understanding
of the errata process and its place within the wider standardisation
process.

Finally, 14.2\% of errata are \emph{reported} but unverified.  Again, we
are surprised to see unverified errata from over a decade ago, suggesting
the process should be expedited. 


%..................................................................................................
\pb{Errata per RFC Area, Status, and Stream:}
Figure~\ref{fig:errata_per_rfc} shows a CDF of the number of errata filed,
per RFC, in each IETF area. Non-IETF RFCs, e.g., IRTF and independent stream
RFCs, and legacy IETF RFCs, are labelled as ``None''.  We confirm errata in
standards are common: of the 4,373 standards-track RFCs in our dataset,
32.7\% have attracted at least one erratum filing.  However, there are
three notable outliers:
\begin{enumerate}
  \item RFCs published by the Sub-IP (subip) Area have very few errata,
    with only 5\% of subip RFCs attracting errata filings. This temporary
    area -- established in 2001 and concluded in 2005 -- only published 19
    RFCs, resulting in a far smaller sample than the other areas. For
    comparison, the next smallest area, General (gen), published 39 RFCs.
    \emph{gen} RFCs attract a greater number of errata on average, vs.
    \emph{subip} RFCs, likely due to their broader relevance.  

  \item We see that both the Application (app) and Security (sec) Area
    RFCs are more likely to have errata filed for them than other areas,
    with 35.9\% of Application and 39\% of Security RFCs attracting at
    least one erratum filing.

  \item Table~\ref{tab:errata-stats-area} details the split between
    \emph{technical} and \emph{editorial} errata across each area.  While
    there is broadly an even split, there are areas where one type of
    errata is more dominant. For example, in the Routing (rtg) area, 60.8\%
    of filings are editorial, while in the Applications (app) area, 59\%
    were technical.  It remains to determine why this is the case, and, in
    particular, to establish whether there is something inherent about the
    RFCs published by these areas that makes them more prone to containing
    errata, and to containing one type of errata vs. another. For example,
    in the Routing area, structured notation is frequently used to define
    routing entities; editorial errata are often filed in those
    definitions. Targeting such areas with improved alternate review
    procedures may be beneficial.

\end{enumerate}

\begin{figure}
  \centering
  \includegraphics[width=\figureWidthOneColumn]{figures-prev/tma-2023/errata-by-rfc-by-area.pdf}
  \caption{
    Cumulative distribution of errata filed per RFC by grouped by IETF
    area.  The areas are as described in Figure~\ref{fig:rfcs_by_area}.
  }
  \label{fig:errata_per_rfc}
\end{figure}

Tables~\ref{tab:errata-stats-stream} and \ref{tab:errata-stats-status}
further categorise errata by the stream and status, at the time of
publication, of each RFC. As expected, the majority of errata are filed
against IETF RFCs and \emph{Proposed Standards} since these make up the
majority of RFCs that are published. However, there are notable differences
in the average number of filings per RFC. \emph{Proposed Standards} (1.01
errata per RFC), \emph{Draft Standards} (1.93), and \emph{Internet
Standards} (2.17) attract a far higher number of errata per RFC than
\emph{Informational} (0.52) or \emph{Experimental} (0.39) documents. This
may be due to the additional readership and attention that standards-track
documents receive, and because they are more likely to be the basis for
future work and protocol extensions.

\begin{table*}
\centering
\footnotesize
\begin{tabular}{lr|rrrr|rr}
\toprule
\textbf{Stream} & \textbf{\#} & \textbf{Verified} & \textbf{Held} & \textbf{Rejected} & \textbf{Reported} & \textbf{Technical} & \textbf{Editorial} \\
\midrule
IETF (6619) & 5797  & 2348  & 1815  & 798  & 836  & 3034  & 2763 \\
IAB (124) & 55  & 25  & 13  & 3  & 14  & 23  & 32 \\
Independent (376) & 330  & 235  & 32  & 28  & 35  & 172  & 158 \\
Legacy (1929) & 510  & 226  & 182  & 39  & 63  & 198  & 312 \\
IRTF (97) & 67  & 37  & 8  & 9  & 13  & 44  & 23 \\
\midrule All & 6759  & 2871  & 2050  & 877  & 961  & 3471  & 3288 \\
\bottomrule
\end{tabular}
\caption{Errata statistics by stream; the ``\emph{Editorial}'' stream has no documents, and is not shown.}
\label{tab:errata-stats-stream}
\end{table*}

\begin{table*}
\centering
\footnotesize
\begin{tabular}{lr|rrrr|rr}
\toprule
\textbf{Status} & \textbf{\#} & \textbf{Verified} & \textbf{Held} & \textbf{Rejected} & \textbf{Reported} & \textbf{Technical} & \textbf{Editorial} \\
\midrule
Proposed Standard (4084) & 4142  & 1680  & 1308  & 555  & 599  & 2213  & 1929 \\
Informational (2894) & 1500  & 719  & 399  & 136  & 246  & 754  & 746 \\
Internet Standard (147) & 319  & 118  & 111  & 66  & 24  & 135  & 184 \\
Best Current Practice (316) & 233  & 111  & 53  & 30  & 39  & 80  & 153 \\
Historic (70) & 20  & 13  & 4  & 2  & 1  & 9  & 11 \\
Draft Standard (142) & 274  & 94  & 93  & 66  & 21  & 150  & 124 \\
Experimental (563) & 221  & 115  & 65  & 20  & 21  & 121  & 100 \\
Unknown (929) & 50  & 21  & 17  & 2  & 10  & 9  & 41 \\
\midrule All & 6759  & 2871  & 2050  & 877  & 961  & 3471  & 3288 \\
\bottomrule
\end{tabular}
\caption{Errata statistics by status at publication.}
\label{tab:errata-stats-status}
\end{table*}


% \todo{Should we include ``Impact of Citations'' from TMA 2023?}

%..................................................................................................
\pb{Errata Location:}
Finally, we investigate the location of errata within RFCs.
Figure~\ref{fig:errata_location_percent_count} presents the number of
errata occurring at each decile of the documents, for the 2,552 filings
where accurate location information is available, and after the copyright
notice and other boilerplate has been removed.  We see that technical
errata dominate over editorial in almost all places, except for the very
beginning where the \textit{Introduction} is located. Moreover, it shows
that the most technical errata are near the middle of the document where
the most complex content is.  We explore where errata occur, with
Figure~\ref{fig:errata_section_wise_counts} showing section titles for
errata appearing in at least $10$ documents. Sections such as the
\emph{Introduction} or \textit{References} are dominated by editorial
errata while more technical sections, like \emph{IANA Considerations} (i.e.,
parameter registrations), \emph{Security Considerations} or
\emph{Definitions}, have a larger proportion of technical errata. In
addition, we see that sections labelled \emph{Appendix} attract a
significant proportion of technical errata. While appendices vary in their
content, they are widely used to provide pseudocode and test vectors, or to
describe algorithms. This suggests that it may be useful to target review
efforts on appendices and other dense technical content where errata are
more likely.

\begin{figure}
  \centering
  \includegraphics[width=\figureWidthOneColumn]{figures-prev/tma-2023/location-percent.pdf}
  \caption{
    Errata counts by percentile location in document (0 is the beginning;
    100 is the end).
  }
  \label{fig:errata_location_percent_count}
\end{figure}

\begin{figure}
  \centering
  \includegraphics[width=\figureWidthOneColumn]{figures-prev/tma-2023/section-title-counts.pdf}
  \caption{
    Errata counts by section title for the more frequent section titles.
  }
  \label{fig:errata_section_wise_counts}
\end{figure}

%--------------------------------------------------------------------------------------------------
\subsection{Summary}

On RFC Production and Complexity (\S\ref{sec:trends-documents:production})
we see evidence that Internet standards are maturing and that the pace of
innovation has slowed. The rate of standards publication peaked in the
mid-2000s, and has gradually declined since, and the number of IETF working
groups has also started to decline. The standards that are being produced
are taking longer, with more revisions prior to publication, and make
increasing use of normative language and references to prior work; all
evidence of complexity and the burden of backwards compatibility. The signs
are that the IETF has developed, and is maintaining and extending, the core
Internet standards, but has not broken out of that niche to new topic areas.


The RFC errata (\S\ref{sec:trends-documents:errata}) process seems to be
broadly effective at finding problems, although there are a concerning
number of unresolved errata reports that are perhaps indicative of an
organisation that doesn't effectively maintain supposedly completed work.
The high prevalence of errata in security-related RFCs also deserves
further scrutiny to determine whether it is due to a higher number of
defects in these RFCs or because defects are more likely to be noticed
because these RFCS receive more review.


%==================================================================================================
\section{Trends in Participant Demographics}
\label{sec:trends-demographics}

In this section, we look at the authorship of published RFCs to explore
how participation in the IETF has changed over time. We show that the
geographic distribution of RFC authors has shifted over time, with the
proportion of authors from the US decreasing and a corresponding rise in
authors from Europe and China (\S\ref{sec:trends-demographics:geographic}).
We similarly explore shifts in affiliation, showing increasing influence of
Huawei and ongoing shifts in the fortunes of many prominent US and European
technology companies (\S\ref{sec:trends-demographics:affiliation}). The
ongoing influence of Cisco and strong input from academic institutions is
also noted.

% The following is from our IMC 2021 paper, Section 3.2.

%--------------------------------------------------------------------------------------------------
\subsection{Geographic Distribution of Authors}
\label{sec:trends-demographics:geographic}

The IETF Datatracker maintains metadata about document authors, including
their names, email addresses, affiliations, and location information. This
dataset does not cover the entire RFC corpus, with metadata available for
authors of RFCs published from 2001, and where it has been provided country
data is available for around 70\% of authors, while affiliation information
is provided for around 80\% of authors. 

Figure~\ref{fig:author_countries_normalised} shows how the proportion of
RFC authors from different countries varies over time, with Figure
\ref{fig:author_continents_normalised} breaking this down by continent.
The IETF has signalled that it wishes to encourage greater geographical
diversity~\cite{RFC7704,ietfblog:diversity}.  Without an explicit goal, we
frame our findings within the context of global population distribution.
We find that North America, while still disproportionately over represented,
is becoming less dominant.  75\% of authors were from North America in 2001,
and this has declined to 44\% in 2020. At the same time, representation of
both Europe and Asia has grown, from 17\% to 40\% and 6\% to 14\%, respectively.
However, Africa and South America remain heavily under-represented, with
only $\approx$0.5\% of authors coming from either continent in 2020.  This
suggests that, if the IETF is to become more geographically representative,
further efforts are needed.

\begin{figure}
  \centering
  \includegraphics[width=\figureWidthOneColumn]{figures-prev/imc-2021/authors/top5_countries_normalised.pdf}
  \caption{
    Authorship countries (normalised)
  }
  \label{fig:author_countries_normalised}
\end{figure}

\begin{figure}
  \centering
  \includegraphics[width=\figureWidthOneColumn]{figures-prev/imc-2021/authors/continents_normalised.pdf}
  \caption{
    Authorship continents (normalised)
  }
  \label{fig:author_continents_normalised}
\end{figure}

%--------------------------------------------------------------------------------------------------
\subsection{Author Affiliations}
\label{sec:trends-demographics:affiliation}

To explore trends in author affiliations, we gather affiliation data from
the IETF Datatracker.  This data is processed to normalise affiliation
names by removing common variations in spelling and to amalgamate known
subsidiaries and merged companies. For example, Huawei and Futurewei are
combined as Huawei, and Sun Microsystems is merged with Oracle.

%..................................................................................................
\pb{Corporate Authors:}
Figure~\ref{fig:author_affiliations_normalised} shows the top ten
affiliations by proportion of RFC authors each year. We observe several
interesting trends.  First, Cisco remains a consistent employer of IETF
contributors, with around 12\% of authors affiliated with the company in
2020, and having been the single largest affiliation across all years in
the dataset. We can also see the rise of Huawei beginning in 2005, with
7.1\% of authors being affiliated with the company in 2020, having peaked
at 9.7\% in 2018. Google has a similar trajectory, first appearing in the
dataset in 2006, with 3.8\% of authors being affiliated with it in 2020.

We also observe the decline of a number of affiliations.  Microsoft and
Nokia, having peaked with 3.3\% and 3.6\% of authors, had 0.7\% and 1.7\%
of authors in 2020, respectively, with the absolute number of authors from
both companies also declining.

These shifts in affiliation reflect changes in fortune and commercial
success of the companies. They also show the ongoing importance of IETF
standards, with new companies choosing to send participants: while we do
not demonstrate any causal link, it is encouraging that commercially
successful companies opt to enable their employees to actively participate
in the IETF. Care is needed to ensure that this relevance is maintained,
however: we observe that the author pool has grown less diverse in terms of
companies that are represented. 35.4\% of authors came from the top 10
affiliations in the dataset in 2020, compared with 25.6\% in 2001. 


\begin{figure}
  \centering
  \includegraphics[width=\figureWidthOneColumn]{figures-prev/imc-2021/authors/top5_affiliations_normalised.pdf}
  \caption{
    Authorship affiliations (normalised)
  }
  \label{fig:author_affiliations_normalised}
\end{figure}

%..................................................................................................
\pb{Academia and consultants:}
Academic affiliations are those where the affiliation name contains
``University'', ``Institute'', or ``College'', while consultancy
affiliations are those that contain ``Consultant`` (we recognise that this
definition of consultants is limited and incomplete). Affiliation data has
been normalised to remove common abbreviations (e.g., ``U.'' for
``University'') and to translate non-English affiliations.  As shown in
Figure~\ref{fig:author_affiliations_normalised}, we find that an increasing
number of authors come from academic affiliations, growing from 8.1\% of
authors in 2001, to 13.6\% in 2020, having peaked at 16.5\% in 2009. The
number of consultants, by our limited measure, has remained stable,
accounting for 2\% of authors in 2020.  The remaining authors are largely
from industrial affiliations.

Figure~\ref{fig:author_affiliations_normalised_acad} shows the top 10
academic affiliations in the dataset, and the percentage of academic
authors that have those affiliations over time. In general, academic
affiliations are each typically held by a small number of authors. We can
see a number of trends in academic authorship, with fewer authors from
Columbia University, MIT, and ISI in recent years, and the rise of Tsinghua
University.

While some academic affiliations are shared by large groups of participants
(e.g., MIT, ISI, Tsinghua University), academic participation is often
driven by a single research group or individual within an institution.
This allows smaller institutions, such as the University of Bremen and
the University of Auckland, that appear in Figure~\ref{fig:author_affiliations_normalised_acad},
and others with single but prolific authors, to have outsize influence.
Understanding whether this indicates academics pushing for standardisation
of the results of government funded research to increase its impact, the
outcomes of industry sponsored research, or academics acting as consultants
for industry, is for further study.

\begin{figure}
  \centering
  \includegraphics[width=\figureWidthOneColumn]{figures-prev/imc-2021/authors/top5_affiliations_normalised_acad.pdf}
  \caption{
    Academic affiliations (normalised)
  }
  \label{fig:author_affiliations_normalised_acad}
\end{figure}

%..................................................................................................
% IMC 2021 paper Section 3.2 (last part)
\pb{Arrival of new authors:}
Figure~\ref{fig:author_new} shows the percentage of authors in each year
that have not previously authored an RFC. Given that the dataset used here
begins in 2001, 100\% of authors are new in that year. The more stable
trend in recent years likely highlights the churn in RFC authorship, with
around 30\% of authors each year having never previously authored an RFC.
This is consistent with the presence of authors with new affiliations
and from new countries and shows an ongoing influx of new participants.

\begin{figure}
  \centering
  \includegraphics[width=\figureWidthOneColumn]{figures-prev/imc-2021/authors/repeat_authors.pdf}
  \caption{
    Percentage of new authors per year.
  }
  \label{fig:author_new}
\end{figure}

%--------------------------------------------------------------------------------------------------
\subsection{Summary}
The trends highlighted in this section  indicate a pool of authors that is
slowly diversifying and changing over time, with a growing proportion of authors
from outside of North America, contributions from new companies and
affiliations, and relatively high authorship churn. Despite this, we find
that certain regions and groups are not well represented in the IETF (e.g.,
contributors from Africa), and that the authorship pool is becoming
increasingly centralised, with a third of authors coming from the top 10
affiliations. This suggests that, if the IETF is concerned with being
more representative, further efforts are needed.


%==================================================================================================
\section{Organisational Dynamics of the IETF}
\label{sec:org-dyn}

While the IETF does hold regular plenary and interim meetings, much of the
interaction between participants takes place on the public mailing lists of
the various working groups. In the following (\S\ref{sec:org-dyn:participants}),
we characterise the mailing list interactions considering the volume of
discussion, the number of drafts discussed, the duration of individual
participation, and how interactions between participants change over time.

We then consider influence and impact of participants (\S\ref{sec:org-dyn:influence}).
We measure how centralised is the active IETF community and how reliant is
it on a small core of participants; how the most influential participants
behave; how influence, determined by mailing list participation, relates to
wider impact in the IETF; and whether the organisational affiliation of
participants influences adoption of new work.

Finally, we consider the hierarchy and communication dynamics in the IETF
(\S\ref{sec:org-dyn:hierarchy}). We ask whether participation in IETF
leadership roles is growing or becoming more centralised onto a small group
of participants; whether the formal organisational hierarchy is reflected
in the patterns of communication between participants; how people in
different roles communicate and whether information flows up or down the
hierarchy; and whether contact with people in leadership positions
influences mobility within the IETF.

%--------------------------------------------------------------------------------------------------
\subsection{Characterising Participant Interactions}
\label{sec:org-dyn:participants}

% IMC 2021 paper Section 3.3

The operations of the IETF is largely underpinned by mailing list
interactions, which are used to discuss and finalise the drafts that
eventually become RFCs. Our data, which starts from 1995, confirms their
vital role, with 1,153 mailing lists, containing 2,439,240 emails from
74,646 unique email addresses.

\pb{Volume of Discussion:}
Figure~\ref{fig:pid_count_emailing_yearly} shows the number of emails sent
across the last 25 years (dashed red line), showing that email volumes have
grown significantly with time, plateauing at around 130,000 messages per
year since around 2010. The figure also shows (solid blue line) how the
number of unique participants (with entity resolution performed as
discussed in Section \ref{sec:background:data}) observed every year in the
mail lists, showing a decreasing trend in the number of contributors since
2007. We note that that the number of participants is broadly correlated
with the rate of RFC publication (Figure \ref{fig:rfcs_by_area}), likely
indicating that many participants disengage once the document they were
working on is published.

\begin{figure}
  \centering
  \includegraphics[width=\figureWidthOneColumn]{figures-prev/imc-2021/emails/pid_count_emailing_yearly.pdf}
  \caption{
    Number of email messages per year and number of unique participants
    exchanging emails per year.
  }
  \label{fig:pid_count_emailing_yearly}
\end{figure}

Figure~\ref{fig:emailvol_by_year_catg} breaks down the messages sent into
four categories based on the sending address:
\begin{enumerate}
  \item \emph{Datatracker-mapped} addresses represent those participants
    sending email in a given year that have an IETF Datatracker account
    associated with their email address. Datatracker accounts are needed to
    submit drafts and register for meetings, and to allow those in
    leadership roles to perform administrative actions, so this indicates
    the fraction of participants who are strongly engaged with the IETF.

  \item \emph{Automated} addresses are those sent by automated systems
    rather than directly by participants. This includes the automated
    announcement of new draft submissions and meeting, last calls for
    comments, etc., sent by the Datatracker, but there is also a growing
    number messages sent by GitHub and other version control systems used
    for managing drafts.
    There were 122 active IETF working groups at the time of writing.
    Of these, 17 listed a GitHub repository in their metadata. The QUIC
    working group, as one example, has replaced the typical email list
    discussions with GitHub issues: indeed, this is a significant part of
    the surge observed in 2016.  Given that most working groups use mailing
    lists to manage their activity, we do not further analyse interactions
    that take place on GitHub. 

  \item \emph{Role-based} addresses represent those sent by people in
    formal leadership or administrative roles in the context of their role.
    There are a very limited number of these addresses, e.g., the IETF,
    IRTF, and IAB Chairs, the IETF Executive Director. As seen, these
    addresses are used sparingly. Note that working group chairs and area
    directors use their individual email addresses when managing their
    working group or area, and so appear as Datatracker mapped addresses.

  \item \emph{Unmapped} addresses are those that do not fall into the other
    categories. They typically represent participants that lurk on a
    mailing list and provide occasional input, but who are not document
    authors and do not attend meetings. This category has reduced over time
    as more types of interaction require a Datatracker account and as
    comments from occasional contributors are increasingly submitted via
    issue trackers on GitHub and similar services.

\end{enumerate}

The increasing use of GitHub suggests that the data in
Figure~\ref{fig:pid_count_emailing_yearly} likely understates the volume of
interactions. The plateau observed in the number of messages sent in recent
years is at least somewhat attributable to the shift to GitHub and similar
services. As this shift continues, it will become important for future work
to consider these interactions.

\begin{figure}
  \centering
  \includegraphics[width=\figureWidthOneColumn]{figures-prev/imc-2021/emails/frequency_emails_yearly_categories2.pdf}
  \caption{
    Number of messages exchanged per year across different categories:
    messages mapped to the Datatracker (Datatracker Person-ID); messages
    by automated email addresses (Automated); messages by role-based
    addresses (Role-based); and messages not mapped to the Datatracker
    (New Person-ID)
  }
  \label{fig:emailvol_by_year_catg}
\end{figure}

%..................................................................................................
\pb{Discussion of draft documents:}
The mailing lists are, other than plenary or interim meetings, the primary
means for the discussion of draft documents. To measure how often drafts are
discussed, we identify mentions of Internet-Drafts and RFCs in mailing list
messages.  We extract any mention of a draft (beginning \texttt{draft-}) or
RFC (i.e., ``RFC'' followed by a number).  Figure~\ref{fig:draft_mentions}
presents the number of drafts mentions in the emails per year. Separate
mentions of the same draft are counted as different mentions, as we want to
observe the entire volume of mentions.  We observe a strong increase in the
number of mentions over time.  This is largely driven by the growing number
of drafts being published. In fact, we find a Pearson correlation of 0.89
between the number of drafts published and the number of mentions. This
speaks to the influence of emails that mention drafts in driving draft
production.
\todo{Add and cross-reference a discussion of number of drafts published
over time in Section \ref{sec:trends-documents}. Does this tie into the
discussion of increasing complexity of draft publication?}

\begin{figure}
  \centering
  \includegraphics[width=\figureWidthOneColumn]{figures-prev/imc-2021/emails/yearly-draft-mention-volume.pdf}
  \caption{
    Number of mentions of Internet-Drafts and RFC in the IETF email
    archive, per year.
  }
  \label{fig:draft_mentions}
\end{figure}

%..................................................................................................
\pb{Duration of Participation:}
We now look into the duration of participation, in terms of the number of
years that people actively participate in any IETF mailing lists.

We define the \textit{contribution duration} of a participant as the length
of time that they have contributed to the mailing lists.  To do this, we
study the participants who first send email to mailing lists between the
years 2000 to 2013. We limit our analysis to 2013, since the longevity of
contributors that first participate more recently cannot be determined.
For each year, in the period 2000 to 2013, we look at those who first
sent email in each year and the number of years that they then go on to
remain active in the mailing lists. For example, a participant who first
sent an e-mail to an IETF list in 2010, and last sent an e-mail to an IETF
list in 2018, will have a contribution duration of 9 years.

First, to understand duration of participation, we first generate Gaussian
Mixture Models for observing different clusters of the maximum duration of
contribution. These models identified three broad clusters into which the
participants can be categorised: \emph{young contributors} for whom the
time between their first and last emails to the IETF lists in one year or
less; \emph{mid-age contributors} who go onto remain active participants in
the email lists for more than one year, but less than five years; and
\emph{senior contributors} who remain participants in the mailing list for
five or more years.

Next, to begin to understand how contribution duration relates to RFC
authorship, we look at the email interactions of the authors of each RFC.
Specifically, we look at the distribution of the contribution duration of
the authors of RFCs, considering the \emph{mean contribution duration} of
all of the authors of the RFC at the time of publication, the contribution
of the \emph{junior-most author} (the author with the lowest contribution
duration at the time of publication), and the contribution of the
\emph{senior-most author} (the author with the highest contribution
duration at time of publication.  Figure~\ref{fig:age_dist_rfc_authors}
shows the distribution of contribution duration of each of these three
measures. This shows that the majority of junior-most authors have
participated for less than 5 years in the IETF (i.e., they are young or
mid-age contributors), whereas the majority of senior-most authors are
senior contributors who have participated in the IETF for significantly
more than 5 years (in fact, 35\% of authors exceed 15 years of IETF
participation).  This shows that RFCs tend to be authored by a mix of
seniority levels.

% We look at outgoing and incoming interactions with RFC authors in the
% period between the first draft and publication of the RFC. If this period
% is less than two years, we look at the activities of authors for two
% years before the RFC was published.  Interactions are defined from the
% viewpoint of the RFC authors: \emph{outgoing interactions} where an RFC
% author responds to an email from other contributors (i.e., email sent);
% and \emph{incoming interactions} where contributors respond to an email
% by the author (i.e., email received).


\begin{figure}
  \centering
  \includegraphics[width=\figureWidthOneColumn]{figures-prev/imc-2021/emails/age_authors_RFCs.pdf}
  \caption{
    Contribution duration distribution of authors of RFCs:
    junior-most author of each RFC, senior-most author of
    each RFC, and mean contribution-duration of all authors
    of each RFC
  }
\label{fig:age_dist_rfc_authors}
\end{figure}

\begin{figure}
  \centering
  \includegraphics[width=\figureWidthOneColumn]{figures-prev/imc-2021/emails/rfc_authors_degree.pdf}
  \caption{
    CDF showing drift in annual degree (interaction with their network)
    of RFC authors over the period 2000-2020.
  }
  \label{fig:degree_dist_rfc_authors}
\end{figure}

%..................................................................................................
\pb{Evolution of interactions:}
The analysis in Section \ref{sec:trends-documents:production} showed that
RFCs are taking longer to publish and show evidence of increasing complexity.
Such increased complexity should be visible in the email discussion during
preparation of the RFCs.

Figure~\ref{fig:degree_dist_rfc_authors} explores this, showing the change
in the number of other people RFC authors interacte with (i.e., the annual
degree of their communication graph) over time. The amount of interactions
RFC authors engage in has substantially increased over time. For instance,
in the year 2000 only 5.5\% of the authors had a degree of over 25 (i.e.,
they exchanged with more than 25 other people), whereas by the year 2015
almost a quarter of the authors had a degree over 25. This confirms that,
on average, more recent RFCs generate greater discussion. This helps to
explain increasing publication times: as documents become more complex,
authors spend more time interacting with other participants to resolve
the complexity, and these discussions are likely to lead to more drafts.

\begin{figure}
  \centering
  \includegraphics[width=\figureWidthOneColumn]{figures-prev/imc-2021/emails/junior_senior_authors_seniormem_indegree.pdf}
  \caption{
    CDF showing number of senior contributors sending messages (in-degree)
    to junior and senior authors.
  }
  \label{fig:junior_senior_indegree_seniormem}
\end{figure}

We also contrast the interaction patterns of the junior vs.\ senior
authors. Figure~\ref{fig:junior_senior_indegree_seniormem} presents the
CDFs of the in-degree (i.e., number of messages received) across both
junior and senior authors. It shows that the incoming interactions from
senior contributors to junior authors are significantly less than the
incoming interactions from senior contributors to senior authors. Nearly
55\% of junior authors receive messages from fewer than 10 senior
contributors, whereas nearly 65\% of senior authors receive messages from
more than 10 senior contributors.  This speaks to the differing roles
played by these sub-populations: senior authors act as hubs through which
substantial volumes of interaction flow.

%..................................................................................................
\pb{Summary:}
Emails play a vital role in underpinning RFC publication with approximately
130,000 emails sent per year. We have shown that the seniority of both
participants and RFC authors fundamentally changes the volume of
interactions that they have. These trends are likely to have implications
for the IETF community, especially as it tries to encourage new
participants.



%--------------------------------------------------------------------------------------------------
\subsection{Characterising Influence and Impact}
\label{sec:org-dyn:influence}

% ICWSM 2022 paper
% - Measuring influence
% - Behaviour of influential participants
% - Impact of influential participants

% This is modified from the Introduction to the ICWSM 2022 paper:

As we showed in Section \ref{sec:org-dyn:participants}, protocol
standardisation is an inherently social process with most day-to-day work
happening on public mailing lists, aided by meetings, video conference
calls, and open document and code repositories.  We are specifically
interested in better understanding how influence is distributed across
stakeholders and how it might affect the standardisation process.  This is
of critical societal importance: the IETF has a major impact on global
Internet technologies, and understanding the social processes involved
would give us insight into not only the driving forces behind
standardisation, but also its resilience to the loss of major participants.
In this section we consider the following research questions: 
\begin{enumerate}
  \item How centralised is the active IETF community, and to what extent
    is it reliant on a small core of participants? 
  \item How do the most influential participants behave? 
  \item How does influence (determined by mailing list participation)
    relate to wider impacts throughout the IETF?
  \item Does the organisational affiliation of participants also
    influence the innovation (adoption of new work) within IETF?
\end{enumerate}

% - - - - - - - - - - - - - - - - - - - - - - - - - - - - - - - - - - - - - - - - - - - - - - - - -
% ICWSM 2022 paper Section 3
\subsubsection{Measuring Influence: The IETF as a Social Graph}

\label{subsec:measuring_influence}

Much of the interaction between IETF participants occurs on public mailing
lists.  For each year, we build a social graph based on the \emph{active
community} of participants (nodes) who have interactions (edges) with any
other participant in the previous 5 years. 

In Section \ref{sec:org-dyn:participants}, we demonstrated that there
are three categories of participants in IETF - (a) \textit{young
contributors} -- who leave within one year of their first year mailing list
contribution; (b) \textit{mid-age contributors} -- participants who stay
active for up-to 5 years; and (c) \textit{senior contributors} -- who
remain active for more than 5 years. Following this, a 5 year period window
is chosen to observe interactions. There are no direct
participant-to-participant email exchanges in our dataset: the emails and
responses we capture are sent to the public mailing lists, and we observe
an interaction between two participants when one replies to an email sent
by another on any mailing list. This yields a social graph based on
1,049,793 emails (out of 2.1M) from 22,138 unique participants across 840
mailing lists.

%..................................................................................................
% ICWSM 2022 paper Section 3.1
\pb{Participation}
To examine the variation in participation between IETF participants, in
Figure~\ref{fig:degree_cdf} we plot the cumulative degree distribution of
the social graph -- i.e., the cumulative number of people a participant
interacts with.  We observe a core group that always interacts with
substantially more people than the rest of the community, with around 80\%
of the participants having degree  $<$15, but 5\%-10\% having degree $>$40.
However,  this difference decreases over time: in 2005-2009 around 6\% of
participants had degree $>$40, but by the period 2015-2019 this increases
to $\sim10\%$.  This shows that while there is always a core of more active
participants, the degree of participation has spread  out over time.
\todo{Repetitive with Section \ref{sec:org-dyn:participants}?}

\begin{figure}
  \centering
  \includegraphics[width=\figureWidthOneColumn]{figures-prev/icwsm-2022/degree_CDF.pdf}
  \caption{
    Cumulative degree distribution of the email graph for different
    year-periods.
  }
  \label{fig:degree_cdf}
\end{figure}

To understand the structure of the community, and its reliance on specific
groups, we analyse the connected components of this graph. Each connected
component reflects a maximal set of nodes such that each pair of nodes is
connected by a path. We compute the size of the Largest Connected Component
(LCC) and the Number of Connected Components (NCC) for each year in
Figure~\ref{fig:lcc_size_yearly}.  The NCC peaks in 2003 before declining,
and broadly aligns with the variation in the number of meeting
attendees,\footnote{\url{https://datatracker.ietf.org/stats/meeting/overview/}}
suggesting a lag between participating for the first time and integrating
into the wider community.  In contrast, the size of the LCC increases until
2006 before stabilising.  Overall, we observe that the IETF community has
become less fragmented.

\begin{figure}
  \centering
  \includegraphics[width=\figureWidthOneColumn]{figures-prev/icwsm-2022/lcc_yearly_no_components.pdf}
  \caption{
    Size of Largest Connected Component (LCC) and Number of Connected
    Components (NCC)
  }
  \label{fig:lcc_size_yearly}
\end{figure}

%..................................................................................................
% ICWSM 2022 paper Section 3.1
\pb{Influence}
Relying on a small group to interconnect the community could undermine the
resilience of the IETF.  To study the influence of participants and their
role in interconnecting the wider community, we compute the betweenness
centrality of each participant in a given time
period~\cite{kourtellis2013identifying, weitzel2012measuring,
sole2014centrality}. We also considered other graph based influence metrics
such as eigenvector centrality: eigenvector centrality reflects the
importance of a node as per its neighbours, while betweenness centrality is
based on shortest paths which is independent of the influence of
neighbours.

However, we found a very strong correlation between the two measures, 
in line with similar experiments \cite{valente2008correlated,he2016correlation}:
the Spearman's rank correlation of participants ranked by betweenness
centrality and eigenvector centrality ranged between 0.51-0.72, with a
strong statistical significance ($p < 0.01$) for the period 2000-2019. We
therefore use just betweenness centrality in our analysis; this is often
studied and acknowledged as a measure of influence in social and complex
networks, particularly built on online communication
\cite{hagen2018crisis,chen2012identifying, ghalmane2019centrality}.

Figure~\ref{fig:purge_nodes_largest_connected_component_number_components}
then shows the effect of removing the most influential participants (in 1\%
increments from most to least, moving left-to-right on the x-axis) on the
size of the LCC.  Worryingly, we find that removal of just 20-25\% of the
most influential participants causes the LCC to shrink by 90\%.  However,
we also find that this impact has decreased over time: for instance, in
2000-2004, removing the top $\sim$5\% influential participants, reduces the
size of LCC by more than half, whereas in 2015-2019, it takes the removal
of the top $\sim$15\% of participants to have the same effect.  This shows
that the community has become \emph{more} cohesive and resilient over time,
and the IETF can now sustain a larger amount of churn while maintaining a
well-connected social graph.

\begin{figure}
  \centering
  \includegraphics[width=\figureWidthOneColumn]{figures-prev/icwsm-2022/lcc_proportion_granular_yearly.pdf}
  \caption{
    Impact of removing participants by their influence on the size
    of the LCC.
  }
  \label{fig:purge_nodes_largest_connected_component_number_components}
\end{figure}

% - - - - - - - - - - - - - - - - - - - - - - - - - - - - - - - - - - - - - - - - - - - - - - - - -
% ICWSM 2022 paper Section 3.2
\subsubsection{Behaviour of Influential Participants}
\label{subsec:influencer_behaviour}

\todo{This is somewhat repetitive with Section \ref{sec:org-dyn:participants}}

We now characterise the behaviour of the most influential participants, in
terms of the volume of emails sent, length of time active within the
community, and topics discussed. 

%..................................................................................................
\pb{Email volume}
Figure~\ref{fig:emailcount_top_percentile} shows that each participant in
the top 10\% most influential participants sends on average around
0.05\%-0.08\% of total emails in a given period.  Collectively, the top
10\% most influential participants account for 43.75\%, on average, of the
total emails, a substantially larger proportion than the others.

This dynamic seems stable over time, making differences even more acute.
At the same time, the overall number of emails sent increases up to 2010
and remains roughly stable from then onward.  Along with the results from
\S\ref{subsec:measuring_influence}, this shows a worrying, if slowly
improving, dependence of the IETF on a small number of influential
participants.

\begin{figure}
  \centering
  \includegraphics[width=\figureWidthOneColumn]{figures-prev/icwsm-2022/lineplot_proportion_emailscount_top_percentile.pdf}
  \caption{
    Proportion of emails sent (\%) by participants in each period (with
    95\% confidence interval) according to their betweenness-centrality
    percentile (x-axis).  Y2-axis (purple) shows yearly count of emails.
  }
  \label{fig:emailcount_top_percentile}
\end{figure}

%..................................................................................................
\pb{Cross-area review} 
In addition to sending more emails, we test if influential participants
engage with different parts of the community more. The work of the IETF is
divided into several areas (e.g., Applications \& Real-time, Security, and
Routing). This eases administration, but might also act as a barrier to
broader discussion and review. For example, WebRTC standards developed in
the Applications and Real-time area might contain elements that could
benefit from expertise from the Security area, but participants in one area
might not review work in another. 

Figure~\ref{fig:areas_participation_top_percentile} shows the mean number
of areas where IETF participants are active (derived from the mailing lists
they use).  We see that influential participants engage in more areas of
the IETF, on average, and that cross-area engagement has improved over
time, indicating that the community itself has matured. This indicates that
influential participants benefit the IETF in enabling cross-area discussion
and review, and can bridge administrative divisions.

\begin{figure}
  \centering
  \includegraphics[width=\figureWidthOneColumn]{figures-prev/icwsm-2022/lineplot_avg_areas_top_percentile.pdf}
  \caption{
    Mean number of areas participated in (with 95\% confidence interval)
    according to their betweenness-centrality percentile in the x-axis,
    ranked from top (0-10) to bottom percentile (91-100).
  }
  \label{fig:areas_participation_top_percentile}
\end{figure}

%..................................................................................................
\pb{Participation duration}
We next consider for how long participants remain associated with the IETF
(measured as the delta between first and last emails sent).
Figure~\ref{fig:age_top_percentile} shows the mean participation duration
distribution for participants, ranked by influence.  Participation duration
increases over time, with the most influential participants typically being
those who have been active for longer. While this might be expected, it
might be a good sign for the IETF community in that it implies that the
most influential participants are also the most experienced.

\begin{figure}
  \centering
  \includegraphics[width=\figureWidthOneColumn]{figures-prev/icwsm-2022/lineplot_age_top_percentile.pdf}
  \caption{
    Mean participation duration (with 95\% confidence interval) according
    to their betweenness-centrality percentile in the x-axis, ranked from
    top (0-10) to bottom percentile (91-100).
  }
  \label{fig:age_top_percentile}
\end{figure}

We extend this observation by considering what happens to participants once
they become influential. Figure \ref{fig:heatmap_top10_overlap} shows what
proportion of the top 10\% influential participants in any given year
(\emph{y-axis}) was also among the top 10\% influential participants in any
other year (\emph{x-axis}). This shows that a significant majority of
participants that become influential, continue to be influential for a
number of years: the top 10\% are influential for at least 6-7 years on
average.

\begin{figure}
  \centering
  \includegraphics[width=\figureWidthOneColumn]{figures-prev/icwsm-2022/heatmap_top10_percent_overlap.pdf}
  \caption{
    Yearly overlap (\%) of top 10\% influencers.
  }
  \label{fig:heatmap_top10_overlap}
\end{figure}

Further, Figure \ref{fig:rate_new_influential_participants} shows that
breaking into the top 10\% of influencers requires an increasing number of
years of participation. This may be beneficial, showing that the IETF is
maturing and is capable of retaining influential and experienced
participants, but it may also point to an increasingly ossified structure
that is not welcoming to newcomers.

\begin{figure}
  \centering
  \includegraphics[width=\figureWidthOneColumn]{figures-prev/icwsm-2022/new_influencers_top10_age_longevityInTop10.pdf}
  \caption{
    Percentage of new influential participants breaking into top 10\%
    influencers (black); the average years of participation before entering
    top 10\% (purple); and minimum \#  years new influencers remain in top
    10\% (with 95\% confidence intervals).
  }
  \label{fig:rate_new_influential_participants}
\end{figure}

%..................................................................................................
\pb{Topics of Discussion} 
The topics discussed on WG mailing lists are a good indicator of the focus
of technical work, and it might be expected that influential participants
will set the direction of that work.  To explore this, we use the Latent
Dirichlet Allocation (LDA) \cite{blei2003latent,hoffman2010online} model
from \texttt{gensim} \cite{rehurek_lrec} to induce 100 topics on the entire
set of email texts. 
Each topic is a distribution over words: e.g., a \emph{security} topic might
have high probability for \emph{cypher}, \emph{rsa}, \emph{auth}, and
related words. We can also use the model to obtain a vector for any input
text as a sparse distribution over all topics, e.g., finding that a text is
comprised of 30\% security and 70\% video streaming topics. With this, we
generate a vector for each participant in a given time period by
concatenating all messages sent by that participant in the period and
feeding it into the LDA model. The result can be loosely interpreted as a
distribution of topics on which each participant works.

Manual inspection of the induced topics reveals some interesting trends. We
find that topics related to \emph{routing} and \emph{email} protocols are
seeing a steady decline in popularity, while topics like \emph{streaming}
and \emph{cryptography} are becoming more prominent. This reflects wider
trends in standardisation, as these efforts have adjusted to the public's
increased awareness of privacy, especially in light of the Edward Snowden
leaks \cite{RFC7258}. We also find that as videoconferencing became
increasingly popular, so did the WebRTC protocol that frequently underpins
them.

We next measure the topical diversity of a participant by observing the
entropy of their topic vector, their \emph{topic entropy}, defined as
follows:
\begin{equation}
  \eta = - \sum_{i=1}^{|T|} p_i \log p_i \nonumber
\end{equation}
where $T = [p_1, ..., p_N]$ is a topic vector, which defines a probability
distribution over $N$ topics, each $p_i$ is a probability of the $i$-th
topic and $\sum_i p_i = 1$.

Figure \ref{fig:topic_diversity} shows topic entropy distributions of
participants in different time periods and influence percentiles. While we
initially experimented with measuring diversity by simply counting the
number of topics that account for the majority ($ \geq 95\% $) of a
participant's topic distribution probability mass, we found that most
participants engage in a relatively small number of topics, regardless of
influence. However, after measuring diversity using entropy, we identify
that the activity of the more influential participants tends to be more
evenly spread across the topics they participate in\footnote{An example
with 3 topics: a distribution of $[0.8, 0.1, 0.1]$ is less evenly spread
than $[0.3, 0.4, 0.3]$.}. This difference becomes more pronounced over
time, implying that increased topic entropy (i.e.,~more evenly participating
in different topics) is an increasingly salient property of influencers.
This is aligned with our earlier findings on cross-area review, showing the
growing number of areas in which influencers participate.


\begin{figure}
  \centering
  \includegraphics[width=\figureWidthOneColumn]{figures-prev/icwsm-2022/topic-div.png}
  \caption{
    Mean topic entropy (with 95\% confidence interval) according to their
    betweenness-centrality percentile in the x-axis, ranked from top (0-10)
    to bottom percentile (91-100).
  }
  \label{fig:topic_diversity}
\end{figure}


% - - - - - - - - - - - - - - - - - - - - - - - - - - - - - - - - - - - - - - - - - - - - - - - - -
\subsubsection{Impact of Influential Participants}
\label{subsec:impact_influence}

% ICWSM 2022 paper Section 3.3

While these top participants are influential within the \emph{mailing
lists}, it is unclear how this influence translates into document
authorship and into gaining  leadership roles within  the organisation.

%..................................................................................................
\pb{Draft authorship}
The output of the IETF is technical standards and other documents published
in the RFC series. RFCs are developed by WGs from a sequence of Internet-Drafts,
with individuals acting as named authors.  Having identified the most
influential participants in the mailing list community, we can determine
whether or not these same participants are also the most active authors.

Figure~\ref{fig:drafts_initiation_top_percentile} shows the distribution of
the proportion of documents authored by mailing list participants, sorted
by their influence rank. We observe a growing proportion of the community
involved in draft authoring, skewed towards influential participants who
tend to write more drafts.  During 2010-2019, each participant in the top
10 percentile, authored 0.125\% to 0.175\% of the total number of drafts in
that period. We find that 32\%-42\% of the total drafts, in different years
during this period, were authored by the top 10\% most influential mailing
list participants.

\begin{figure}
  \centering
  \includegraphics[width=\figureWidthOneColumn]{figures-prev/icwsm-2022/documents_proportion_published_top_percentile.pdf}
  \caption{
    Average drafts authored (with 95\% confidence interval) according to
    their betweenness-centrality percentile in the x-axis, ranked from top
    (0-10) to bottom percentile (91-100).  Right-y axis (purple) shows
    yearly drafts per year.
  }
  \label{fig:drafts_initiation_top_percentile}
\end{figure}

%..................................................................................................
\pb{Co-authorship and email graph correlation}
Figure~\ref{fig:drafts_initiation_top_percentile} showed that influential
mailing list participants tend to author more drafts than others.
We next ask if they are also influencers in the draft co-authorship graph.

Thus, we create a draft co-authorship graph where each author is a node,
and draft co-authorship is an edge. We then measure influence with
betweenness centrality of the authors in each time period.
Table~\ref{tbl:correlation_sprmn} compares the top 20\% of influencers of
the mailing lists with those of the co-authorship graph.  We find a
significant overlap between both groups, ranging from 48.2\% to 67.2\%.
There is a significant ($p < 0.05$) positive correlation between the
rankings of the overlapping members of each community: participants that
are influential in the mailing lists are also likely to be influential in
draft authorship.

\begin{table}
  \small 
  \begin{tabular}{llllll}
    \toprule
    \multicolumn{6}{c}{Top 20\% draft authors \& all email participants}\\ \midrule
                    
    Years   & \multicolumn{2}{c}{Sub-Network Size} & Overlap  & \multicolumn{2}{c}{Spearman's $r_{s}$} \\ \cline{2-3}  \cline{5-6}
        &   Co-author & Email  &  &   $r_{s}$   & $p$-value \\ \midrule
        
    2000-04   &   398 &   1390 &   48.24\%  &  .323 &   4.75e-06 \\
    2005-09   &   427 &   1662 &   67.21\% &   .332   &   7.39e-09    \\
    2010-14   &   728 &   1639 &   63.05\% &   .299   &   5.73e-11  \\
    2015-19   &   915 &   1370 &   55.85\%  &   .337   &   4.13e-15  \\ %\hline
       
    \bottomrule
  \end{tabular}
  \caption{
    Overlap in the co-author and email graph.
  }
  \label{tbl:correlation_sprmn}
\end{table}

We also look at the top 20\% participants from the co-authorship network
who are \emph{not} part of the top 20\% influencers in the email network in
Figure \ref{fig:non_overlapping_coauthornetwork_authors_emailnetwork}.
We observe that not all the prolific authors are that engaged in the email
discussion: 40\%-50\% of non-overlapping authors are ranked between 20th to
40th percentile of influence in the email network. These non-overlapping
authors are typically more junior with respect to their participation
duration.

\begin{figure}
  \centering
  \includegraphics[width=\figureWidthOneColumn]{figures-prev/icwsm-2022/coauthor_affiliation_network/lineplot_non_overlapping_authors_emailnetwork_percentile.pdf}
  \caption{
    Percentage of Top 20\% authors that are not 20\% influencers according
    to their betweenness-centrality (email graph) percentile in the x-axis,
    ranked from  top (0-10) to bottom percentile (91-100).
  }
  \label{fig:non_overlapping_coauthornetwork_authors_emailnetwork}
\end{figure}

%..................................................................................................
\pb{Leadership roles} 
WG chairs are selected from the community, and we might expect those
selected to be influential in the community.
Figure~\ref{fig:wg_chair_before_after} shows that this is indeed the case:
87.5\% of WG chairs are in the top 20\% of mailing list influencers and
67.4\% are in the top 20\% of document authors in the year before they
became chairs. This also shows the impact of taking up a leadership role:
influence in both the mailing list and authorship communities grows in the
year after participants become a WG chair for the first time. 


\begin{figure}
  \centering
  \includegraphics[width=\figureWidthOneColumn]{figures-prev/icwsm-2022/WG_chair_percentile_CDF_before_after.pdf}
  \caption{
    CDF of the percentile of betweenness-centrality of WG chairs in the
    email and co-authorship graph, one year before and after becoming
    chairs.
  }
  \label{fig:wg_chair_before_after}
\end{figure}

%..................................................................................................
\pb{Organisations}  While participants in the IETF contribute as
individuals, they are usually affiliated with an organisation.  To study
the potential influence of organisations, for each draft we obtain the
authors' affiliations from the Datatracker. If this is not available, we
use the domain name of the authors' email addresses, which we map to the
relevant organisation (e.g., \texttt{@cisco.com} maps to Cisco). For generic
email addresses, such as \texttt{@gmail.com}, we use the participant name
if no affiliation is available.

Figure \ref{fig:coauthor_affiliation_network2019_otheryears} shows the ten
most frequent affiliated organisations of the top 20\% authors in the
period 2015-2019, and the number of authors affiliated with these
organisations in  each  period.  While the dominance of Cisco is clear,
other organisations, such as Huawei, have gained a larger presence over
time.  In general, a small number of organisations employ a significant
fraction of the influential participants in the standards process. If we
look at the period 2015-2019, for example, out of the 915 authors ranked in
the top 20\% of most influential authors, 342 belong to the ten most
frequently affiliated organisations, and nearly 253 are from Cisco, Huawei,
Ericsson, or Juniper alone.

\begin{figure}
  \centering
  \includegraphics[width=\figureWidthOneColumn]{figures-prev/icwsm-2022/coauthor_affiliation_network/2019_affiliations_in_otheryears_coauthor_network.pdf}
  \caption{
    Authors in the most common affiliations of top 20\% authors
    (co-authorship graph) in 2015-2019.
  }
  \label{fig:coauthor_affiliation_network2019_otheryears}
\end{figure}

While authors from the same organisation often co-author drafts together,
collaborations between authors from different organisations are also
common.  Figure~\ref{fig:heatmap_2019_coauthordraft_most_frequent_org}
shows the most common collaborations between authors from the top 15 most
frequently affiliated organisations.  Collaborations between authors at
competing organisations, such as between authors from Cisco, Huawei,
Ericsson, and Juniper, occur frequently. We also computed that over the
years, joint collaboration (authors from multiple affiliations) has
increased: 772 co-authored drafts in the period 2000-2004 vs. 3083
co-authored drafts in the period 2015-2019.

\begin{figure}
  \centering
  \includegraphics[width=\figureWidthOneColumn]{figures-prev/icwsm-2022/coauthor_affiliation_network/heatmap_2019_top10_affiliation_collaboration.pdf}
  \caption{
    Number of drafts co-authored by top 15 most frequently affiliated
    organisations. We only include the top 20\% authors (co-authorship
    graph) of 2015-2019.
  }
  \label{fig:heatmap_2019_coauthordraft_most_frequent_org}
\end{figure}

Further inspection of the collaboration trends reveals that the volume of
collaborations between authors from Huawei and other organisations appears
to be unaffected by the recent addition of Huawei to the U.S. Entity List
\cite{bis:2019:entity} as the trends in Figure~\ref{fig:heatmap_2019_coauthordraft_most_frequent_org}
also hold for 2020-2021.\footnote{\tiny\url{https://mailarchive.ietf.org/arch/msg/ietf-announce/0ywjgSS4LlO0DaWDoLJLRHxJdUk/}}

% - - - - - - - - - - - - - - - - - - - - - - - - - - - - - - - - - - - - - - - - - - - - - - - - -
\subsubsection{Summary}

% ICWSM 2022 paper section 4.3

We conclude that the IETF is still reasonably centralised, but that this
has improved over the years. However, removing around 20\%-25\% most
influential participants still fragments the entire network
(\S\ref{subsec:measuring_influence}, Figure
\ref{fig:purge_nodes_largest_connected_component_number_components}). The
most influential people are the ones with highest degree of engagements
within the community who also tend to be more involved in draft authoring
activities (\S\ref{subsec:influencer_behaviour}, Figure
\ref{fig:drafts_initiation_top_percentile}). However, we also observe that
over the years more people (even the ones less influential in the email
network) are getting involved in draft authoring activities (Figure
\ref{fig:drafts_initiation_top_percentile}). This shows that these
activities, while centralised to a certain degree, are still open to
contributions from the wider IETF community.

We observe that the influential participants show a much higher level of
engagement with the community (\S\ref{subsec:influencer_behaviour}). A
considerable proportion of total emails is sent by the top 10\% of
influential participants (Figure \ref{fig:emailcount_top_percentile}).
Compared to the rest of participants, they are active in more areas (Figure
\ref{fig:areas_participation_top_percentile}), are active within the IETF
for a longer period of time (Figure \ref{fig:age_top_percentile}), and
participate in a more diverse set of topics (Figure
\ref{fig:topic_diversity}). Finally, their influence extends from the
mailing lists to other activities such as draft authorship.

A significant overlap and correlation is observed between
influential authors from co-authorship network and email networks
(\S\ref{subsec:impact_influence}, Table \ref{tbl:correlation_sprmn}). This
shows that a large set of authors exhibit an ability to co-author drafts as
well as engage with participants on the email networks, thereby translating
their influence from email networks to co-authorship and vice versa. 

A considerable portion of influential participants ($\sim$30\%) are
affiliated to one of the more prominent organisations (e.g., \emph{Cisco}
or \emph{Ericsson}) (\S\ref{subsec:impact_influence}, Figure
\ref{fig:coauthor_affiliation_network2019_otheryears}). Participants from
different organisations do considerably collaborate (Figure
\ref{fig:heatmap_2019_coauthordraft_most_frequent_org}). Moreover, the
level of such collaboration is found to have increased with time. 




%--------------------------------------------------------------------------------------------------
\subsection{Hierarchy and Communication Dynamics}
\label{sec:org-dyn:hierarchy}

% ICWSM 2024 paper
% - RQ1: Are higher levels of the hierarchy growing or becoming more
%   centralised? Are they associated with an increased domination of the
%   conversation? 
% - RQ2: What is the association between the organisation hierarchy and
%   general communication patterns?
% - RQ3: How do people with differing roles communicate with each other?
%   Does information flow “up” or “down” the hierarchy? 
% - RQ4: What is the impact that individuals have on their direct contacts’
%   activity over time? How does this vary across hierarchy level?

In the following, we quantitatively determine the effect that organisational
hierarchy levels have on communication patterns throughout the IETF. 
% We model communication as a network where a node represents an individual
% and a directed edge represents a communication sent from one individual to
% another~\cite{panzarasa2009patterns,viswanath2009evolution,klimt2004enron}.
% By repeating the analysis across time, we can build a full temporal
% network~\cite{masuda2016guide} of communication interactions between people
% in the system.In this study we consider only the network structure and the
% status at that time of the individuals, not the content of messages. 
We study how the email communication is impacted by the hierarchical roles
of individuals. According to the IETF mission statement~\cite{rfc3935}, the
IETF is made up of volunteers who collaborate to develop consensus-based
technical standards. This collaboration should be evident in their
communication practises.

%..................................................................................................
% ICWSM 2024 paper section 2

In the following we consider the following research questions: 
\begin{enumerate}
  \item
    Are higher levels of the hierarchy growing or becoming more
    centralised? Are they associated with an increased domination
    of the conversation?
    (\S\ref{sec:org-dyn:hierarchy:rq1})

  \item
    What is the association between the organisation hierarchy and general
    communication patterns?
    (\S\ref{sec:org-dyn:hierarchy:rq2})

  \item
    How do people with differing roles communicate with each other?  Does
    information flow ``up" or ``down" the hierarchy?
    (\S\ref{sec:org-dyn:hierarchy:rq3})

  \item
    What is the impact that individuals have on their direct contacts'
    activity over time? How does this vary across hierarchy level?
    (\S\ref{sec:org-dyn:hierarchy:rq4})

\end{enumerate}

The first hypothesis focuses on the impact of the
``steepness"~\cite{anderson2010functions} of the hierarchy. This refers to
the fewer people in higher hierarchy levels the more steep the hierarchy
is, i.e a centralised control of decision making in higher levels. The
opposite of this is a ``diffused"~\cite{hussain2019voice} hierarchy where
the large amount of people in higher levels stifle lower level voice by
creating ``bystanders" who find it hard to speak up. In general, the
conclusions about steep hierarchies are confused, where some papers
conclude a positive and some a negative effect of steep hierarchies. We
explore what this means for the IETF in RQ1.

The second hypothesis regards the ``power
distance"~\cite{li2021does,duan2018authoritarian,guo2020inclusive} of
individuals, which is the number of hierarchy levels between people who are
communicating. The focus is on the impact this has on people's voice or
lack thereof when the distance is large versus small. The conclusions from
this research area usually suggest that even small power distances have a
large effect of reducing the voice of the lower of the two levels. We
tackle this for the IETF in three different ways outlined in RQ2-4.

The IETF being a voluntary organisation gives a new perspective on these
hypotheses, as the existing literature focuses on commercial organisations.
In contrast to the above conclusions, we demonstrate that despite the
hierarchy becoming more ``diffused", participants in the higher levels in
the IETF hierarchy perform a ``facilitator" role, promoting discussion with
those they interact with and receiving a benefit themselves as a result. We
also show that higher level participants tend to be a focus of discussion
with remarks directed ``upward" toward them whereas Regular Participants
(RPs) engage more in group-discussion with less focus on an individual.
This suggests that the IETF does not conform to the notion that power
distance has a negative effect on communication, nor that a more diffused
hierarchy reduces the voice of lower levels.

% - - - - - - - - - - - - - - - - - - - - - - - - - - - - - - - - - - - - - - - - - - - - - - - - -
\subsubsection{Methodology}
\label{sec:org-dyn:hierarchy:methodology}

% From our ICWSM 2024 paper, section 4:



% - - - - - - - - - - - - - - - - - - - - - - - - - - - - - - - - - - - - - - - - - - - - - - - - -
\subsubsection{Centralisation and Dominance}
\label{sec:org-dyn:hierarchy:rq1}

% RQ1: Are higher levels of the hierarchy growing or becoming more
% centralised? Are they associated with an increased domination of
% the conversation?

%..................................................................................................
% From our ICWSM 2024 paper, Section 2:

The effect the steepness of an organisation's hierarchy has on the
communication within is not well understood. This RQ is important because
once we know whether the IETF hierarchy is becoming steeper, we can look at
the effects this is having on the communication patterns of participants.

In this RQ, we look at the number of individuals and/or roles that inhabit
each hierarchy level. We also consider whether individuals in higher levels
are more active in the conversation and quantify the evolution of the
mailing list activity within each level of the IETF hierarchy. We find that
the middle level, Working Group Chairs (WGCs), is gaining in overall
proportion of activity, whilst the lowest level, RPs, is decreasing. The
top level, Area Directors (ADs), remains constant in proportion. The
increase in WGC activity coincides with an increase of WGCs per Working
Groups (WGs).

%..................................................................................................
% From our ICWSM 2024 paper, Section 5.1:

For \textbf{RQ1}, to determine the steepness or diffusion of the hierarchy,
the proportion of nodes and activity are determined in each hierarchy
level. Figure~\ref{fig:activity_proportions}a shows the number of active
nodes in each level. The bottom level, RPs, are the largest contingent
consisting of 90--86\% of active participants; this proportion is
decreasing over time. The next level is WGCs which manage the WGs and are
6--10\% of active participants; this proportion is rising over time. The
top level is ADs which manage many WGs within an area and consist of less
than 1\% of the total active population of the IETF; this stays constant
over time. Therefore, the WGC level is becoming more diffused over time as
there are about 15 RPs for every one WGC in the organisation in 2013 versus
8 RPs per WGC in 2021. 

The proportion of activity of nodes in the last year is also plotted and
split into hierarchy level in Figure~\ref{fig:activity_proportions}b. It is
clear that individuals at higher levels contribute a disproportionate share
to communication and this share is increasing over time, which is
consistent with findings in \cite{khare2022web}. Therefore, less of a
proportion of communication is flowing from the RPs at the same time as the
number of WGC roles is increasing. This further suggests that the IETF
hierarchy is becoming more diffused, in terms of mailing list
communication, which may make cooperation between hierarchy levels more
difficult.

\begin{figure}[t]
  \centering
  \small{a)} \\
  \includegraphics[width=\figureWidthOneColumn]{figures-prev/icwsm-2024/active_node_proportions_year_sliding_window.png} \\
  \small{b)} \\
  \includegraphics[width=\figureWidthOneColumn]{figures-prev/icwsm-2024/activity_proportions_year_sliding_window.png}
  \caption{
    Proportions of RPs, WGCs and ADs by a) number of active individuals and
    b) activity (number of emails sent to mailing lists). 
  }
  \label{fig:activity_proportions}
\end{figure}

Figure~\ref{fig:wgc_over_time}a shows the amount of nodes that inhabit the
WGC hierarchy level over our time period. There is a $35\%$ rise in the
amount of WGC roles whereas the amount of individuals that fill the WGC
roles remains mostly constant.

Figure~\ref{fig:wgc_over_time}b contains two estimates of the number of WGs
over time. The estimated number of WGs is likely the most accurate in the
middle 50\% of both plotted lines. Their discrepancy is the least (at most
10 WGs) from late 2014 to early 2019. The start of the COVID-19 pandemic
may also affect mailing list activity in 2020, and similar trends are
visible in the data in \cite{mcquistin2021characterising}. A full
explanation on how the data is estimated can be found in
section~\ref{sec:WG_activity}.

Figure~\ref{fig:wgc_over_time}c shows how many WGCs exist per WG, using
both estimates of the WG amount from Figure~\ref{fig:wgc_over_time}b. It is
clear there is a slow a rise in WGCs per WG over time. The ratio rises from
between 1.7-1.9 to 2.0-2.2 within the middle 50\%, late 2014 to early
2019.This perhaps represents a growing understanding in the IETF that it is
desirable to have two or more chairs per WG in case of
illness/unavailability or to avoid conflicts of interest. Therefore, this
plot again shows that the hierarchy is becoming more diffused, even if the
raw amount of individuals in the WGC roles has remained constant. The
multiple roles per individual may result in even more difficulty of RPs to
use their voice.

An AD version of Figure~\ref{fig:wgc_over_time} is available, but it is
omitted to save space as all three plots stay mostly constant throughout
the period. There are about 15 individual ADs split amongst 15 roles in 7-8
areas with about 2 ADs per Area. The set of IETF Areas is broadly fixed,
while WGs are created and closed relatively frequently.


\begin{figure}[t]
  \centering
  \includegraphics[width=\figureWidthOneColumn]{figures-prev/icwsm-2024/over_time.png}
  \caption{
    Overview of WGC numbers and WG numbers over time. a) shows the number
    of WGC roles (some individuals hold multiple WGC roles) and the number
    of individuals who are WGCs b) shows two different estimates of the
    number of WGs (see section~\ref{sec:method}) and c) shows the mean
    number of WGCs per WG. Notice the non-zero-based y-axis exaggerates the
    variation in these figures.
  }
  \label{fig:wgc_over_time}
\end{figure}

An AD version of Figure~\ref{fig:wgc_over_time} is available, but it is
omitted to save space as all three plots stay mostly constant throughout
the period. There are about 15 individual ADs split amongst 15 roles in 7-8
areas with about 2 ADs per Area. The set of IETF Areas is broadly fixed,
while WGs are created and closed relatively frequently.

Table~\ref{tab:WGC_email} shows the emails sent and received for WGC and
ADs on mailing lists in the year before they first take that role, and the
year after. WGCs and ADs are both active in using the mailing lists, ADs
much more so. However, they don't become more active to any major degree
after taking the roles. This may suggest that the individuals who take on
higher hierarchy level roles are already contributing at a higher level.

\begin{table}[!tbp]
\centering
    \begin{tabular}{c||c|c}
        Status & Sent & Received \\
         \hline \hline 
        Before WGC & $67.26\pm99.80$ & $70.11\pm103.37$ \\
        \hline
        After WGC & $77.20\pm102.85$ & $85.90\pm107.38$ \\
        \hline
        Before AD & $218.84\pm226.92$ & $230.50\pm214.75$ \\
        \hline
        After AD & $209.21\pm186.46$ & $219.16\pm189.85$ \\
    \end{tabular}
        \caption{Mean number of emails received for one year before/after becoming a WGC (AD) for the first time. The error is one standard deviation.}
    \label{tab:WGC_email}
\end{table}

\pb{Summary}
In answer to RQ1 the picture is mixed, the number of individuals with WGC
roles has remained broadly constant but the number of WGC roles has
increased, indicating that individuals willing to become WGC have taken on
more such roles. The number of the higher level AD roles has, by design,
remained broadly constant over the time studied. The proportion of active
individuals with WGC roles has grown from 6\% to 10\% and the proportion of
the conversation taken by those roles has grown from 25\% to 35\% over the
period studied. Therefore the hierarchy has become more diffused, which may
mean RPs experience more difficulty in expressing their voice.


% - - - - - - - - - - - - - - - - - - - - - - - - - - - - - - - - - - - - - - - - - - - - - - - - -
\subsubsection{Organisational Hierarchy and Communication Patterns}
\label{sec:org-dyn:hierarchy:rq2}

% RQ2: What is the association between the organisation hierarchy and
% general communication patterns?

%..................................................................................................
% From our ICWSM 2024 paper, Section 2:

The communication patterns that different levels in the hierarchy
experience may be  indicative of the cooperation between
levels~\cite{li2021does,duan2018authoritarian}. For instance, if
participants at higher levels send lots of email while receiving little,
this suggests a lack of receptiveness of higher levels to the suggestions
of lower levels, and low confidence of lower levels to voice their
opinions. In contrast, we hypothesise that collaboration between layers is
high, as outlined in the IETF mission statement, which will look more like
the opposite of this example.

In this vein, we compare the ways that different roles communicate, looking
at the tendency for communications to be purely inbound, outbound or use
more complex patterns. We count three edge motifs for different hierarchy
levels as described in section~\ref{sec:motif}. We find that higher levels
consistently experience a higher proportion of incoming communication than
RPs, and all levels send outward communication at similar proportions over
time. Also, the proportion of people who begin email threads are split by
hierarchy level. Higher levels show a disproportionate tendency to
originate email threads. 

%..................................................................................................
% From our ICWSM 2024 paper, Section 5.2:

To quantify the effect of being in each hierarchy level has on network wide
communication for \textbf{RQ2}, we calculate the proportion of three node
motifs, lasting at most a month, in which each node participates over a
year period, see section~\ref{sec:motif}. These motifs are counted,
categorised into the three hierarchy levels and then proportions are taken
of each motif type. 

Figure~\ref{fig:motif_proportions} shows the proportions of each level for
three node ``Inward Star", ``Outward Star", ``Mixed Star", and
``Triangles". The Inward and Mixed Star motifs show about a $4\%$ and
$10\%$ increase in proportion for higher levels in the hierarchy versus for
RPs. However, about a $10\%$ increase for RPs is seen for Triangles versus
WGCs and ADs, and Outward Star proportions remain similar for all levels of
the hierarchy.

The larger proportion for WGCs and ADs of Inward and Mixed Star motifs
whilst the Outward Star motifs remain similar for all levels suggests that
higher levels receive more direct communication. The higher proportion of
RP Triangle motifs suggests their discussion is more of a group activity,
whereas WGCs and ADs have a larger proportion of one-on-one conversations.
Therefore, we interpret the general communication patterns within the IETF
as a discussion amongst RPs interspersed with questions sent to and
announcements from WGCs and ADs. This indicates some collaboration between
layers and confidence of some RPs' to voice their opinions upwards.

\begin{figure*}[t]
  \centering
  \includegraphics[width=\figureWidthTwoColumn]{figures-prev/icwsm-2024/motif_strata_comparison.png}
  \caption{
    Communication patterns analysed by three edge temporal motifs for RPs,
  }
  \label{fig:motif_proportions}
\end{figure*}


Figure~\ref{fig:origin_emails} seems to corroborate this. We calculate the
proportion of mailing list threads which nodes in each hierarchy level
originate, using a year window, pushed forward by a month each calculation.
We see that WGCs and ADs send a disproportionate number of originating
emails to the mailing lists. WGCs send 30--50\% whereas they are 6--10\% of
active individuals, and ADs 5--10\% versus 0.5\%. 

Therefore, the larger proportion of inward motifs for higher levels is in
part due to WGCs and ADs disproportionately originating email threads.
Other IETF participants will then reply to the thread, boosting the
proportion of inward motifs WGCs and ADs appear in. This may suggest the
higher levels are good ``condensers of communication" as they receive more
than they send.

In answer to the RQ, combining the interpretations of the two plots, the
discussion seems to happen in the following way: The ADs or WGCs are
commonly the originators of discussion threads. Where conversation includes
a WGC or AD the conversation tends to be directed around them and usually
inward towards the WGC or AD. The RPs are more likely to engage in
discussions amongst themselves (triangular communication patterns) when
higher level individuals are not present. 

\begin{figure}[t]
  \centering
  \includegraphics[width=\figureWidthOneColumn]{figures-prev/icwsm-2024/Origin_email_proportions.png}
  \caption{
    Proportion of ``origin" (non reply) emails sent by RPs, WGCs and ADs,
    this should be interpreted in conjunction with
    figure~\ref{fig:activity_proportions} showing the proportion of active
    individuals in each class.
  }
  \label{fig:origin_emails}
\end{figure}

% - - - - - - - - - - - - - - - - - - - - - - - - - - - - - - - - - - - - - - - - - - - - - - - - -
\subsubsection{Communication up and down the Hierarchy}
\label{sec:org-dyn:hierarchy:rq3}

% RQ3: How do people with differing roles communicate with each other?
% Does information flow ``up" or ``down" the hierarchy?

%..................................................................................................
% From our ICWSM 2024 paper, Section 2:

Our third hypothesis is that the cooperative nature of the IETF will be
evident in the effect that power distance~\cite{li2021does} has on
communication throughout the organisation. The raw proportion of the
communication between hierarchy levels helps us to understand the
confidence of lower levels to exercise their voice. This proportion, for a
cooperative and voluntary organisation should be higher for lower hierarchy
levels, showing a confidence in their voice.

We categorise edges based on the hierarchy levels they originate from and
are sent to. Ratios are then calculated of ``upward" versus ``downward"
communication between RPs, WGCs and ADs. We find that the lower the level,
the more of a skew to upward communication.  Communication tends to flow
``up'' the hierarchy more than ``down''. 

%..................................................................................................
% From our ICWSM 2024 paper, Section 5.3:

For \textbf{RQ3}, we determine the direction communication tends to flow
through the hierarchy by looking at inbound and outbound communication
flows. The network's edges are categorised based on their source and target
node's hierarchy level. For each of the inter-level communication
combinations the proportion of communication in the last year going in the
``upward" hierarchy direction is calculated. For instance, if there are six
emails from RPs to WGCs and twelve from WGCs to RPs then the proportion
upward is $\frac{6}{6+12}=\frac{1}{3}$. 

In Figure~\ref{fig:inter_strata}, these proportions are plotted, and the
year long window is again pushed forward by one month to increase the
resolution of changes. A proportion of higher than $0.5$ shows more
communication flowing up the hierarchy than down. All three levels are
mostly upward in direction, with the lowest two levels (RP$\rightarrow$
WGC) having the largest skew in the proportion. The other two lines
(RP$\rightarrow$AD and WGC$\rightarrow$AD) show similar periodicity, with
the WGC$\rightarrow$AD line dropping as far below $0.5$ as above showing a
large shift in inter-level communication patterns higher up the hierarchy.

The direct measurement of message flow between the hierarchy levels seems
to bolster our interpretation that the cooperative design of the IETF is
working. This analysis shows that RPs are communicating much more
preferentially upwards in the hierarchy, despite the decrease in overall
share of RP activity. The increase within the WGC level does not have any
noticeable effect on their level of communication, if anything there is a
negative relationship. This again suggests both WGCs and ADs perform a kind
of ``condenser of communication" or ``facilitator" role. In other words,
this analysis suggests the IETF has a ``bottom-up" communication style,
where the higher levels encourage the lower levels' voice.  This is the
desired pattern of communication for a volunteer organisation that is
looking to grow and encourage participants to engage and gradually take on
leadership roles.

Overall for RQ3 then, the pattern observed was communication flowing up the
hierarchy and this corresponds to the answer found in RQ2. WGCs and ADs
were more likely to receive communication with individuals than send
communication to individuals whereas both were more likely than RPs to send
out messages to the list in general. 

\begin{figure}[t]
  \centering
  \includegraphics[width=\figureWidthOneColumn]{figures-prev/icwsm-2024/up_hierarchy_communication_ratios.png}
  \caption{
    Proportion of communications that are ``up" the hierarchy for different
    groups. A proportion above 0.5 indicates that the majority of
    communications are ``upward".
  }
  \label{fig:inter_strata}
\end{figure}

% - - - - - - - - - - - - - - - - - - - - - - - - - - - - - - - - - - - - - - - - - - - - - - - - -
\subsubsection{Mobility Within the Hierarchy}
\label{sec:org-dyn:hierarchy:rq4}

% RQ4: What is the impact that individuals have on their direct contacts'
% activity over time? How does this vary across hierarchy level?

%..................................................................................................
% From our ICWSM 2024 paper, Section 2:

An important question~\cite{li2021does,guo2020inclusive} for any
organisation is whether higher level participants encourage those at lower
levels in the hierarchy to properly engage. The IETF is cooperative and
voluntary by design, therefore we hypothesise a high level of encouragement
from higher levels to lower. We measure whether activity in one time period
is associated with activity in a subsequent time period both for
individuals and for neighbourhoods. If those who communicate with higher
levels receive a boost in their communication as a result, this is an
indication of such encouragement.

In particular, we perform Mobility Taxonomy analysis to determine the
effect individual IETF participants have on their future number of
connections, and the average number of connections of their neighbours.  We
find that WGCs have an increased tendency, versus RPs, to remain active
(\textit{Mobility}), an increased tendency for their neighbours to be
active subsequent to high WGC activity (\textit{Philanthropy}) and for them
to gain activity after their neighbours are active (\textit{Community}).
We interpret this as WGCs filling a ``facilitator" role on the mailing
lists.


%..................................................................................................
% From our ICWSM 2024 paper, Section 5.4:

Finally, to determine how individuals and their neighbours affect each
other's activity levels in subsequent time periods, for \textbf{RQ4}, we
take correlations of degree between different time windows using the
Mobility Taxonomy~\cite{barnes2023measuring} aspects called
\textit{Mobility}, \textit{Neighbour Mobility}, \textit{Philanthropy} and
\textit{Community} explained in the previous section. In
Figure~\ref{fig:mob_tax}, the time window chosen is one year which is split
into two snapshots graphs of six months, the time window then moves forward
one month and the process is repeated. The mid point of the year time
window is plotted. WGCs and RPs are plotted but the small number of ADs
mean that the correlations are too noisy to show meaningful effects. 

\begin{figure*}[t]
  \centering
  \includegraphics[width=\figureWidthTwoColumn]{figures-prev/icwsm-2024/hierarchy_status_comparison.png}
  \caption{
    Connection between roles and activity for RP and WGC viewed in terms of
    a) Mobility b) Neighbour Mobility c) Philanthropy and d) Community over
    time. See section~\ref{subsec:mob_tax} for definitions of these terms.
  }
  \label{fig:mob_tax}
\end{figure*}

The \textit{Mobility} of both RPs and WGCs is plotted in
Figure~\ref{fig:mob_tax}a. The high amount of correlation between the
degree of an individual in the first half of the time window with the
second half shows that all users have a tendency to become more active in
subsequent time periods if they are highly active in the previous period,
but this is more pronounced in WGCs. This is found to follow a period of
about one year for both RPs and WGC, suggesting that a person is likely to
be as active as there were a year ago. The \textit{Neighbour Mobility}
plot, Figure~\ref{fig:mob_tax}b, shows a similar tendency for
neighbourhoods, but with less of a split.

The \textit{Philanthropy} plot, Figure~\ref{fig:mob_tax}c, shows a higher
correlation for WGCs than RPs between an individual's degree in the first
snapshot with and their ND in the second. This means that individuals who
directly interact with WGCs become more active on mailing lists
subsequently. The \textit{Community} plot, Figure~\ref{fig:mob_tax}d, shows
the correlation between the ND in the first snapshot with the degree in the
second.  The higher correlation for WGCs suggests they benefit from
interactions with highly active neighbours by themselves becoming more
active in subsequent time periods. These trends for \textit{Philanthropy}
and \textit{Community} show a reversal in early 2016 that coincides with a
period where the mean number of WGCs per WG and the number of WGC roles was
rapidly rising, see figure~\ref{fig:wgc_over_time}. A merging of the
Applications area and Real-time Applications \& Infrastructure area in May
2015 may be a cause of this reversal of correlation, as the assignment of
WGs to areas was changed which may cause more non-reciprocal WGC
communication than normal.

Increased \textit{Mobility} for WGCs suggests that they gain advantage from
their role with their activity leading to similar communication activity in
the future. Moreover, they experience an increased effect of
\textit{Philanthropy} and \textit{Community}. The former suggests that if a
WGC is engaged in a higher amount of discussion, then in the future the
individuals who they were communicating with will also be highly active.
The latter suggests that when WGCs are surrounded by people who are active
in discussion, they are likely to be more active themselves in the future
too. All of this has less of an effect for RPs who show a lower correlation
in terms of all three aspects. Our interpretation of this is that active
WGCs are ``facilitating" discussion in the WG mailing lists; both having
their activity boosted by their neighbours and, in turn, boosting the
activities of their neighbours. This aligns well with the hypothesis that
the cooperative and voluntary design of the IETF lends itself well to
higher levels encouraging those lower to engage in discussion. 

Overall the answer to RQ4 is that the WGCs have a positive effect on their
immediate contacts and those individuals that directly discuss with WGCs
are more likely to engage more in discussion in subsequent time periods.
Conversely though the WGCs who discuss topics with individuals who are
active in discussion are, themselves, more likely to engage more fully in
discussion. This points to something like a virtuous cycle of WGCs
encouraging and being encouraged by their direct contacts. 

% - - - - - - - - - - - - - - - - - - - - - - - - - - - - - - - - - - - - - - - - - - - - - - - - -
\subsubsection{Summary}

% From our ICWSM 2024 paper, Section 6:

\todo{summarise hierarchy results}

%==================================================================================================
\section{Language and Influence}
\label{sec:language}

% Tracing Linguistic Markers
% - RQ1: How do linguistic traits differ between more and less influential
%   participants? 
% - RQ2: How do linguistic traits vary for participants at different levels
%   of the organisation hierarchy? 
% - RQ3: How does linguistic behaviour of participants change as they gain
%   influence?

% Frontiers in Psychology paper
% - Use of Language, discussing how it shifts as people become more senior

% LEDA
% - a taxonomy of dialogue acts (DAs) and a labeled dataset of emails



%==================================================================================================
\section{Success Factors}
\label{sec:success-factors}

\subsection{What Makes a Successful Author?}
% ICWSM 2022 paper Section 4


% ICWSM 2022 paper section 4.3
The statistical analysis shows that influence of draft authors in the email
networks does impact the possibility of a draft getting adopted
(\S\ref{subsec:results}, Table \ref{tbl:res} \& \ref{tbl:resstat}). This
might hint at the ability of participants who hold a domain expertise to be
able to engage better with the community. Several WG chairs are already in
the top percentile influential category in both the email and co-authorship
networks before taking up these leadership roles, which further elevates
after taking up such leadership roles (Figure
\ref{fig:wg_chair_before_after}).

Most importantly the statistical analysis shows that being affiliated with
a prominent organisation positively impacts the chances of a draft in
getting adopted by a WG, thereby directly driving the innovation process
(Table \ref{tbl:res} \& \ref{tbl:resstat}).

\subsection{What Makes a Successful Document?}

% IMC 2021 paper Section 4

%==================================================================================================
\section{Related Work}
\label{sec:related}

% This should come near the end, and focussing on discussing how your work
% relates to that of others. Any relevant related work should have been
% cited already, so this is not a list of related work, it's a discussion
% of how that work relates.
%
% Why not put related work after the introduction? 1) because describing
% alternative approaches gets between the reader and your idea; and 2)
% because the reader knows nothing about the problem yet, so your
% (carefully trimmed) description of various technical trade-offs is
% absolutely incomprehensible.
%
% When writing the related work:
%  - Give credit to others where it's due; this doesn't diminish the
%    credit you get from your paper.
%  - Acknowledge weaknesses in your approach.
%  - Ensure related work is accurate and up-to-date



%==================================================================================================
\section{Conclusions}
\label{sec:conclusions}


%==================================================================================================
\section*{Acknowledgements}

% Acknowledge funding sources.

This work has been supported in part by the UK Engineering and Physical
Sciences Research Council under grants EP/S033564/1 and EP/S036075/1
(``Streamlining Social Decision Making for Improved Internet Standards'').

%==================================================================================================
\bibliographystyle{abbrv}
\bibliography{paper}

%==================================================================================================
% The following information gets written into the PDF file information:
\ifpdf
  \pdfinfo{
    /Title        (An Empirical Analysis of the Internet Engineering Task Force with Computational Methods)
    /Author       (...)
    /Subject      (Internet Standards)
    /Keywords     (IETF, Internet, Standardisation)
    /CreationDate (D:20240922164700Z)
    /ModDate      (D:20240922164700Z)
    /Creator      (LaTeX)
    /Producer     (pdfTeX)
  }
  % Suppress unnecessary metadata, to ensure the PDF generated by pdflatex is
  % identical each time it is built. This needs pdfTeX 3.14159265-2.6-1.40.17
  % or later.
  \ifdefined\pdftrailerid
    \pdftrailerid{}
    \pdfsuppressptexinfo=15
  \fi
\fi
%==================================================================================================
\end{document}
% vim: set ts=2 sw=2 tw=75 et ai:
